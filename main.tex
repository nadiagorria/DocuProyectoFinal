% Documentación del Proyecto Final - Estilo visual APA (numerado)
\documentclass[12pt]{article}
% Idioma y codificación
\usepackage[spanish]{babel}
\usepackage[utf8]{inputenc}
\usepackage[T1]{fontenc}

% Tipografía y espaciado (estilo similar a APA)
\usepackage{mathptmx} % Times-like font (compatible)
\usepackage{setspace}
\doublespacing % APA suele usar interlineado doble

% Márgenes APA: 1 pulgada (2.54 cm)
\usepackage[letterpaper,margin=1in]{geometry}

% Encabezados y formato de secciones
\usepackage{fancyhdr}
\usepackage{titlesec}
\usepackage{graphicx}
\usepackage{enumitem}
\usepackage{hyperref}

% Configuración de encabezado (running head)
\pagestyle{fancy}
\fancyhf{}
\fancyhead[L]{\textit{Título corto del trabajo}}
\fancyhead[R]{\thepage}
\renewcommand{\headrulewidth}{0pt}

% Secciones estilo APA: flush left, bold, no extra vertical space
\titleformat{\section}{\normalfont\bfseries\large}{\thesection.}{0.5em}{}
\titleformat{\subsection}{\normalfont\bfseries}{\thesubsection.}{0.5em}{}

\setcounter{secnumdepth}{4}
\setcounter{tocdepth}{3}

\title{Documentación del Proyecto Final (APA visual)}
\author{Nombre del Autor}
\date{\today}

\begin{document}
\maketitle
\vspace{1cm}
\tableofcontents
\clearpage

% Secciones principales (numeradas)

\section*{Agradecimientos}
\addcontentsline{toc}{section}{Agradecimientos}

Queremos expresar nuestro más sincero agradecimiento a aquellas personas que nos han acompañado y apoyado a lo largo de nuestra trayectoria académica. En especial, a nuestro docente y tutor Nicolás Escobar, quien ha sido un pilar fundamental en nuestra formación profesional. Sus observaciones, correcciones y sugerencias nos permitieron mejorar la calidad del proyecto y mantener una dirección clara en el proceso de desarrollo.

Agradecemos también a nuestras familias, por su apoyo constante y por facilitar las condiciones necesarias para llevar adelante este trabajo, especialmente durante los períodos de mayor dedicación. Su comprensión y acompañamiento hicieron posible sostener el ritmo de trabajo requerido.

A todas esas personas, nuestro más sincero agradecimiento.


\section*{Resumen}
\phantomsection
\addcontentsline{toc}{section}{Resumen}
El sistema web integral desarrollado para el Instituto Nacional de Investigación Agropecuaria (INIA) surge como una solución tecnológica para modernizar y digitalizar los procesos de análisis de calidad de semillas. El proyecto, realizado como trabajo final de la carrera Tecnólogo en Informática, aborda la necesidad de unificar y estandarizar procedimientos técnicos que anteriormente se gestionaban mediante planillas Excel.

El sistema permite administrar lotes y diversos análisis, como Germinación, Tetrazolio, PMS (Peso de Mil Semillas), DOSN (Determinación de Otras Semillas en Número) y Pureza Física, incorporando validaciones basadas en estándares internacionales, trazabilidad e historial técnico por lote, normalización de catálogos, importación de datos legados y generación de reportes especializados con exportación consolidada a Excel.

La arquitectura se implementó siguiendo un modelo cliente-servidor en capas, con un backend en Java Spring Boot 3.5 expuesto como API REST y reforzado con seguridad por JWT (RFC 7519) y autenticación en dos factores (2FA). Se incorporaron notificaciones en tiempo real mediante WebSockets, paginación \textit{offset} y \textit{limit}, y pruebas automatizadas con JUnit/JaCoCo. El frontend fue desarrollado en Next.js 14 con TypeScript, ofreciendo una PWA (Progressive Web App) optimizada y con un buen rendimiento.

El sistema resultante mejora la trazabilidad de los datos, la consistencia metodológica de los análisis y la eficiencia operativa institucional, estableciendo una base sólida para futuras ampliaciones y para la interoperabilidad con otros sistemas del INIA.

\section{Introducción}
En la última década, la digitalización ha transformado profundamente la gestión de información en sectores productivos y científicos, impulsando nuevos estándares de eficiencia, trazabilidad y precisión en los procesos operativos. En este contexto, el ámbito agropecuario no ha sido la excepción: la demanda de sistemas capaces de centralizar datos, reducir tareas manuales y garantizar la consistencia metodológica se ha vuelto fundamental para sostener la calidad de los análisis y la toma de decisiones institucionales.

Previo a este proyecto, el proceso de análisis de calidad de semillas en el Instituto Nacional de Investigación Agropecuaria (INIA) presentaba desafíos significativos derivados del uso de registros dispersos, planillas Excel independientes y procedimientos manuales con alto riesgo de error\cite{inia-uy}. Los análisis técnicos —como Germinación, Tetrazolio, PMS, DOSN y Pureza Física— se realizaban sobre lotes de semillas cuyos datos se registraban de forma heterogénea, dificultando el seguimiento del historial completo de cada lote\cite{ista2023,fao-seed}. Esta fragmentación generaba esfuerzos adicionales de normalización y verificación, con impacto en los tiempos de respuesta, en la trazabilidad de los resultados y en la comparabilidad histórica de la información técnica. Además, la ausencia de notificaciones oportunas y la falta de catálogos unificados limitaban la coordinación operativa entre equipos.

Frente a esta necesidad, surgió el desarrollo del Sistema Web Integral INIA, una plataforma diseñada para modernizar, centralizar y estandarizar todo el ciclo de vida de la información vinculada a los análisis de calidad de semillas. El proyecto, realizado como trabajo final de la carrera Tecnólogo en Informática, propone una solución que gestiona de manera estructurada los lotes, sus datos técnicos asociados, los contactos involucrados, las validaciones específicas de cada tipo de análisis, el historial completo de resultados y la generación automatizada de reportes. Asimismo, se incorporó la importación de información proveniente de sistemas legados, preservando datos históricos relevantes y facilitando la transición tecnológica.

En conjunto, esta solución tecnológica permitió profesionalizar la gestión de los análisis y de los lotes de semillas, reducir errores operativos, fortalecer la trazabilidad institucional y mejorar sustancialmente la eficiencia en el registro, control y procesamiento de información técnica. El sistema constituye una base sólida para futuras ampliaciones, nuevos módulos de interoperabilidad y una evolución continua hacia un ecosistema digital integral para el INIA.

\section{Objetivos Planteados y Resultados Esperados}
El principal objetivo del proyecto fue analizar, diseñar y desarrollar un sistema integral para la gestión de análisis de semillas realizados por el Instituto Nacional de Investigación Agropecuaria, orientado a la optimización del proceso actual de registro, edición, consulta y trazabilidad de las muestras (lotes) ingresadas para su análisis. Este sistema busca centralizar y modernizar la operativa ya existente, brindando tecnologías que faciliten la agilizar tareas, reducir errores y mejorar la disponibilidad de información para todos los posibles usuarios.
Durante el proceso de construcción del sistema se evaluaron diferentes alternativas tecnológicas para su implementación, optando por Java 21 junto con Springboot 3.5 y Spring Data JPA para el backend REST. Mientras que en el frontend se consideró React con Tailwind CSS 4.1. Asimismo se buscó la integración de módulos complementarios, como notificaciones, reportes y la exportación e importación de archivos formato xlsx.

\subsection{Objetivos específicos}
\begin{itemize}
  \item Un sistema centralizado y estable para gestionar todo el ciclo de análisis.
  \item Mayor trazabilidad y transparencia en el seguimiento de lotes.
  \item Reducción de errores mediante automatización y validaciones.
  \item Disminución del tiempo dedicado a tareas administrativas.
  \item Operativa más moderna, simple e intuitiva.
\end{itemize}

\section{Estado del Arte}
En los últimos años, la transformación digital se ha vuelto una realidad en prácticamente todas las organizaciones, tanto públicas como privadas. Este proceso busca mejorar la forma en que se gestionan los datos y se llevan adelante las tareas diarias. En el sector agropecuario, especialmente, la digitalización ha demostrado ser clave para lograr una mayor trazabilidad, precisión en los registros y eficiencia operativa. Por este motivo, cada vez más instituciones están dejando atrás los documentos en papel, las planillas de Excel y los procedimientos dispersos, y están migrando hacia plataformas web que centralizan y organizan la información de forma más segura y confiable. Esta tendencia está ampliamente respaldada por estudios académicos y por la práctica común de centros de investigación y laboratorios en todo el mundo.

\cite{inia-uy,ista2023,fao-seed}

El proyecto desarrollado para el INIA se inscribe en esta línea de evolución tecnológica. La institución necesita una herramienta digital que permita gestionar sus análisis de forma integral: registrar y consultar muestras (lotes), administrar configuraciones, controlar distintos roles de usuario, generar reportes, exportar información y también incorporar datos históricos provenientes de sistemas anteriores. Este capítulo presenta el marco conceptual, tecnológico y comparativo que justifica la solución propuesta y contextualiza su diseño dentro de las tendencias actuales de la industria del software.

\cite{inia-uy}
\subsection{Marco Teórico}
En estos últimos años las organizaciones adoptaron hojas de cálculo como herramienta base para almacenar información gracias a su accesibilidad, bajo costo, flexibilidad y facilidad de uso. Sin embargo, a medida que los procesos se vuelven más complejos y los volúmenes de datos crecen, estos mecanismos se vuelven insuficientes y arriesgados para labores críticas. Las limitaciones más comunes incluyen:
\begin{itemize}
    \item \textbf{Escalabilidad restringida:} Las hojas de cálculo no están diseñadas para manejar grandes volúmenes de datos ni crecer de forma sostenible. A medida que aumentan los registros, las operaciones se vuelven lentas, propensas a fallos y difíciles de mantener.
    \item \textbf{Colaboración limitada:} La funcionalidad de edición simultánea que es ofrecida por la mayoría de las aplicaciones de gestión de hojas de cálculo es vulnerable a conflictos, sobreescrituras y pérdida de información.
    \item \textbf{Falta de trazabilidad y mecanismos de auditoría:} Resulta difícil rastrear cambios, identificar responsables y asegurar integridad de datos.
    \item \textbf{Integración deficiente con sistemas externos:} La conexión con APIs, bases de datos, servicios externos u otros sistemas institucionales resulta limitada.
\end{itemize}
\subsection{La Transición Hacia Sistemas Web}
Las planillas suelen funcionar bien en etapas iniciales, pero a medida que crecen los volúmenes de información o los procesos se vuelven más complejos, es común que se busquen alternativas web que permitan centralizar la información y manejarla de manera más eficiente. Los sistemas modernos ofrecen:
\begin{itemize}
    \item Acceso multiplataforma.
    \item Escalabilidad horizontal mediante arquitecturas distribuidas.
    \item Integración nativa a través de APIs REST.
    \item Validaciones automáticas a nivel de negocio.
    \item Auditorías completas de operaciones.
    \item Estandarización de procesos y flujos de trabajo.
\end{itemize}
\cite{ista2023,fao-seed}
Este enfoque tecnológico es el más utilizado en laboratorios, centros de investigación y organizaciones científicas para gestionar muestras, controles de calidad, análisis y trazabilidad de procesos experimentales.
\subsection{Tecnologías}
El proyecto adopta tecnologías modernas, robustas y ampliamente utilizadas en la industria. La selección busca equilibrio entre estabilidad, madurez, facilidad de mantenimiento y alineación con estándares profesionales.
\cite{spring-boot,nextjs}
\subsubsection{Frontend: React + TypeScript}
React es uno de los frameworks más utilizados globalmente para construir interfaces de usuario debido a:
\begin{itemize}
    \item Su modelo de componentes reutilizables
    \item Su alto rendimiento mediante virtual DOM
    \item Un ecosistema amplio de librerías
    \item Facilidad para construir SPA (Single Page Applications)
\end{itemize}
El uso de TypeScript aporta tipado estático, reduce errores y mejora la mantenibilidad del código.
\subsubsection{Backend: Java + Spring Boot}
Spring Boot es estándar en el desarrollo empresarial gracias a:
\begin{itemize}
    \item Integración nativa con Spring Security
    \item Soporte simplificado para APIs REST
    \item Inyección de dependencias y modularidad
    \item Comunidad madura y documentación extensa
\end{itemize}
Java 21, como versión LTS, asegura estabilidad a largo plazo.
\subsubsection{Base de Datos: PostgreSQL}
Elegida por ser:
\begin{itemize}
    \item Open source
    \item Altamente confiable
    \item Compatible con operaciones complejas
    \item Escalable y sólida para manejo de datos institucionales
\end{itemize}
\subsubsection{Infraestructura y Arquitectura}
\begin{itemize}
    \item Modelo cliente-servidor (frontend Next.js – backend Spring Boot)
    \item Arquitectura en tres capas (presentación – servicios – datos)
    \item Uso de Docker para portabilidad y despliegue
    \item Estructura modular por dominios (análisis, seguridad, notificaciones, usuarios)
    \item Comunicación vía REST y WebSocket (notificaciones en tiempo real) \cite{rfc6455,rfc7519}
\end{itemize}
\subsection{Soluciones Similares}
En el mercado existen herramientas orientadas a gestionar información estructurada, colaborar en equipos y reemplazar flujos basados en planillas. Aunque ninguna se adapta exactamente a los procesos complejos del INIA, sirven como referencia sobre cómo la industria resuelve problemas similares.
\subsubsection{Airtable}
Airtable combina conceptos de base de datos con la interfaz amigable de una hoja de cálculo. Es una plataforma low-code orientada a pequeños proyectos y equipos que necesitan digitalizar procesos sin desarrollar software propio.

Ventajas:
\begin{itemize}
    \item Interfaz simple y accesible
    \item Colaboración en tiempo real
    \item APIs integradas
    \item Automatizaciones básicas
\end{itemize}
Limitaciones:
\begin{itemize}
    \item No escala para flujos complejos
    \item Restricciones para reglas de negocio avanzadas
    \item Dependencia de licencias externas
\end{itemize}
\cite{airtable}
\begin{figure}[htbp]
    \centering
    \includegraphics[width=0.8\textwidth,height=0.6\textheight,keepaspectratio]{Airtable.png}
    \caption{Ejemplo: vista de contenido en Airtable (tabla/registro).}
    \label{fig:airtable}
\end{figure}
\subsubsection{Smartsheet}
Smartsheet ofrece una experiencia similar a Excel, pero con mayor control, trazabilidad y herramientas para flujos de trabajo.

Ventajas:
\begin{itemize}
    \item Gestión de proyectos y procesos
    \item Reportes avanzados
    \item Automatizaciones integradas
\end{itemize}
Limitaciones:
\begin{itemize}
    \item Alto costo según uso
    \item Menor flexibilidad frente a un desarrollo a medida
    \item Dependencia del ecosistema propietario
\end{itemize}
\cite{smartsheet}
\begin{figure}[htbp]
    \centering
    \includegraphics[width=0.8\textwidth,height=0.6\textheight,keepaspectratio]{Smartsheet.png}
    \caption{Ejemplo: interfaz de Smartsheet mostrando una vista tipo hoja de cálculo y reportes.}
    \label{fig:smartsheet}
\end{figure}
\subsubsection{Odoo}
Odoo es un ERP (Planificador de Recursos Empresariales) modular que integra distintos dominios empresariales.

Ventajas:
\begin{itemize}
    \item Gran variedad de módulos
    \item Comunidad activa
    \item Escalabilidad para múltiples áreas
\end{itemize}
Limitaciones:
\begin{itemize}
    \item Instalación y configuración complejas
    \item Sobrecarga funcional para proyectos específicos
    \item Dificultad de adaptación a procesos científicos y de laboratorio
\end{itemize}
\cite{odoo}
\begin{figure}[htbp]
    \centering
    \includegraphics[width=0.8\textwidth,height=0.6\textheight,keepaspectratio]{Odoo.png}
    \caption{Ejemplo: módulo de Odoo con vistas integradas de gestión de recursos.}
    \label{fig:odoo}
\end{figure}
\subsection{Conclusión del análisis comparativo}
Si bien las herramientas ofrecen funcionalidades útiles, no cumplen de forma precisa con los requerimientos del INIA, especialmente en lo relativo a:
\begin{itemize}
    \item Procesos de laboratorio
    \item Gestión de análisis y múltiples roles
    \item Validaciones específicas
    \item Importación de datos legados
    \item Adaptación a flujos técnicos
\end{itemize}
Por ello, un sistema a medida es la opción más adecuada para atender las necesidades reales y garantizar la adaptación total a los procesos internos y ofrecer la flexibilidad necesaria para futuros cambios o expansiones.
\cite{inia-uy,ista2023}
\section{Análisis del Problema}
Para poder desarrollar el sistema web del INIA fue necesario entender a fondo cuáles eran los problemas reales en el trabajo diario del laboratorio. Durante mucho tiempo los análisis se registraban en planillas haciendo difícil mantener un control claro y asegurar la uniformidad de la información. Este diagnóstico inicial fue clave para diseñar una solución que realmente mejorase la forma en que los análisis se gestionan y resultase útil para el cliente.
A partir de este análisis, se estudiaron cuidadosamente los procesos actuales, se conversó con los usuarios y se documentaron sus necesidades. Esto permitió definir cómo debía funcionar el sistema, qué información era importante, quiénes interactúan con él y qué herramientas debían construirse para hacer su trabajo más sencillo y seguro. Con esta información se estableció el alcance del proyecto y se identificaron los puntos críticos a resolver en cuanto a diseño y desarrollo.
\subsection{Vista del Modelo de Dominio}
El modelo de dominio reúne todas las entidades del sistema y muestra cómo se relacionan todas entre sí. Tener esta vista clara fue fundamental para ordenar las ideas, entender el flujo de la información y asegurarse de que el sistema reflejase fielmente la forma en que se trabaja en el laboratorio.
\begin{figure}[htbp]
    \centering
    \includegraphics[width=0.8\textwidth,height=0.6\textheight,keepaspectratio]{\detokenize{Diagrama de Dominio.png}}
    \caption{Modelo de dominio: entidades y relaciones principales.}
    \label{fig:modelo_dominio}
\end{figure}
\subsection{Definición de Casos de Uso}
Los casos de uso ayudan a describir qué puede hacer cada usuario y cómo se relacionan las acciones con las funcionalidades disponibles en el sistema. A partir de ellos se identificaron tareas clave como crear y gestionar lotes y análisis, administrar catálogos, generar reportes y cargar datos históricos.
\subsection{Actores}
Los actores son los elementos externos al sistema, ya sean usuarios o sistemas, que interactúan con la plataforma para llevar a cabo determinadas tareas. En este proyecto se definieron tres actores centrales:

\textbf{Administrador:}
\begin{itemize}
    \item Administra catálogos, lista de contactos y aprobación de usuarios.
    \item Gestiona análisis y lotes.
    \item Importa datos históricos.
    \item Otorga y revoca permisos.
    \item Exporta reportes en formato xlsx.
    \item Supervisa y valida las acciones realizadas por los Analistas.
    \item Observa reportes.
    \item Tiene acceso total a todos los módulos del sistema.
\end{itemize}
\textbf{Analista:}
\begin{itemize}
    \item Registra y edita lotes y análisis.
    \item Carga y modifica información técnica.
    \item Requiere la aprobación del Administrador para confirmar los resultados de un análisis.
    \item Exporta reportes en formato xlsx.
    \item Observa reportes.
\end{itemize}
\textbf{Observador:}
\begin{itemize}
    \item Solo puede visualizar información.
    \item Accede a lotes, análisis, resultados y reportes.
    \item No posee permisos para modificar datos.
\end{itemize}
\subsection{Diagrama de Casos de Uso}
El siguiente diagrama muestra los principales casos de uso del sistema:
\begin{figure}[htbp]
    \centering
    \includegraphics[width=0.8\textwidth,height=0.6\textheight,keepaspectratio]{\detokenize{Diagrama Casos de Uso.png}}
    \caption{Diagrama de casos de uso: principales interacciones entre actores y el sistema.}
    \label{fig:casos_uso}
\end{figure}
\subsection{Vista del Modelo de Diseño}
El diseño propuesto para la arquitectura del sistema busca mantener un equilibrio claro entre simplicidad, organización y capacidad de crecimiento. La estructura en capas responde a la necesidad de separar responsabilidades y asegurar que cada parte del sistema pueda evolucionar sin generar impacto innecesario en las demás. Este enfoque no solo mejora la mantenibilidad, sino que también facilita la incorporación de nuevas funcionalidades en el futuro. 
La organización del sistema se basa en una arquitectura por capas que incluye la capa de Cliente, donde se ubica la aplicación web, incluyendo sus capacidades de PWA, y la capa de Presentación, encargada de gestionar la interfaz y la comunicación inicial con el backend. A esto se suma la capa de Seguridad, que centraliza los mecanismos de autenticación, autorización y protección de datos. Por su parte, la capa de Aplicación, implementada con Spring Boot, contiene toda la lógica del negocio y se encarga del flujo de información hacia la capa de Datos, responsable del acceso y persistencia.
En esta arquitectura, los servicios adicionales que utiliza el sistema, como el envío de correos o la verificación en dos pasos, se integran directamente dentro de la capa de Aplicación. Esto evita complejidades innecesarias en la representación del diseño y mantiene el diagrama coherente y simple, sin dejar de reflejar el funcionamiento real del sistema.
\begin{figure}[htbp]
    \centering
    \includegraphics[width=0.8\textwidth,height=0.6\textheight,keepaspectratio]{\detokenize{Diagrama de Arquitectura.png}}
    \caption{Diagrama de arquitectura: capas y componentes principales del sistema.}
    \label{fig:arquitectura}
\end{figure}
\subsection{Descripción de la Arquitectura del Sistema}
La arquitectura propuesta organiza el sistema en capas bien definidas, lo que permite mantener una estructura ordenada, escalable y fácil de mantener. Cada capa cumple un rol específico dentro del flujo general de la aplicación.
\subsubsection{Capa de Cliente (SPA/PWA)}
El frontend funciona como una aplicación web moderna basada en Next.js y diseñada como SPA con capacidades PWA. Desde el navegador, los usuarios interactúan mediante una interfaz rápida, responsiva y adaptable a distintos dispositivos. La aplicación soporta instalación como app y toda la comunicación con el backend se realiza a través de REST.
\subsubsection{Capa de Seguridad}
Incluye los mecanismos que protegen el acceso al sistema. Se utiliza autenticación basada en JWT almacenado en cookies seguras y un sistema de doble factor (2FA) mediante códigos TOTP compatibles con Google Authenticator. La autorización se gestiona con Spring Security mediante roles como ADMIN, ANALISTA y OBSERVADOR. El frontend también aporta seguridad con middleware que controla el acceso a rutas protegidas. Además, existe un sistema de notificaciones por correo para avisos de seguridad y recuperación de cuentas.
\subsubsection{Capa de Presentación (Frontend)}
Esta capa engloba la lógica de presentación, los componentes visuales, la validación de formularios y el manejo de estado. Se utilizan herramientas como React Query, React Hook Form, Zod, Radix UI, shadcn/ui y Tailwind. El frontend organiza sus rutas y funcionalidades en módulos bien definidos: autenticación, administración, análisis, reportes, perfil, notificaciones, etc. También ofrece integración con WebSockets para recibir actualizaciones y notificaciones automáticamente e incluye funcionalidades como dashboards interactivos, listados, formularios complejos e instalación como PWA.
\subsubsection{Capa de Aplicación (Backend)}
El backend está construido con Spring Boot siguiendo el patrón MVC y estructurado en Controllers, Services y Repositories. Los Controllers exponen las APIs, los Services contienen la lógica de negocio y los Repositories gestionan el acceso a la base de datos. Aquí se manejan análisis de semillas, validaciones, reportes, importación de datos históricos, seguridad, notificaciones y todo el flujo del sistema. También se implementa un canal WebSocket con STOMP para notificaciones en tiempo real, manejo de transacciones y un sistema global de manejo de excepciones.
\subsubsection{Capa de Datos}
La persistencia está implementada con Spring Data JPA e Hibernate, usando PostgreSQL como base de datos. La base está diseñada con relaciones bien definidas e integridad referencial.
\subsection{Descomposición en subsistemas}
El sistema se organiza en cinco subsistemas principales que trabajan en conjunto para ofrecer una plataforma robusta, segura y orientada a la experiencia del usuario. Cada uno cumple un rol específico dentro de la solución, y en conjunto conforman una arquitectura coherente y fácil de mantener. A continuación, se detalla cada subsistema y los elementos que lo componen.
\subsubsection{Subsistema de Cliente}
Representa lo que usa el usuario en su navegador o como PWA instalada.
\begin{itemize}
    \item Renderiza la interfaz y maneja la experiencia visual e interacción.
    \item Funciona incluso con conexión inestable
    \item Administra cookies de sesión y comunicación con el backend vía HTTPS y WebSockets.
    \item Se encarga de notificaciones push y la instalación como app en dispositivos.
\end{itemize}
\subsubsection{Subsistema de Seguridad}
\begin{itemize}
    \item Gestiona la autenticación con JWT y Cookies HttpOnly.
    \item Implementa 2FA con TOTP y manejo de dispositivos confiables.
    \item Controla los permisos según roles: ADMIN, ANALISTA, OBSERVADOR.
    \item Administra contraseñas, recuperación de acceso y validaciones.
    \item Incluye middleware que protege rutas y verifica sesiones en el frontend.
    \item Genera alertas por eventos críticos de seguridad (cambios de clave).
\end{itemize}
\subsubsection{Subsistema de Presentación (Frontend)}
\begin{itemize}
    \item SPA/PWA construida con componentes reutilizables.
    \item Renderiza pantallas, formularios, tablas, flujos y navegación interna.
    \item Maneja el estado global de la aplicación.
    \item Canaliza toda la comunicación hacia el backend mediante servicios API.
    \item Recibe actualizaciones en tiempo real mediante WebSockets.
    \item Aplica reglas de visibilidad según rol del usuario.
\end{itemize}
\subsubsection{Subsistema de Aplicación (Backend)}
\begin{itemize}
    \item Contiene la lógica de negocio principal del sistema.
    \item Expone endpoints REST desde los controllers.
    \item Ejecuta reglas de negocio, validaciones e integraciones internas.
    \item Administra transacciones y operaciones complejas de múltiples pasos.
    \item Gestiona módulos como catálogos, reportes, usuarios, análisis, lotes, etc.
    \item Envía notificaciones en tiempo real al frontend.
    \item Devuelve respuestas seguras, consistentes y auditables.
\end{itemize}
\subsubsection{Subsistema de Datos}
\begin{itemize}
    \item Gestiona el acceso a la base de datos PostgreSQL.
    \item Define las entidades del dominio y su estructura.
    \item Implementa repositorios JPA para consultas y persistencia.
    \item Garantiza integridad, consistencia y trazabilidad de la información.
    \item Registra auditoría automática de creación y modificación de registros.
\end{itemize}
\section{Implementación}
\subsection{Diagrama de \textit{Deployment} de UML}
El diagrama de \textit{Deployment} permite mostrar de manera clara cómo quedó distribuida la arquitectura física del sistema una vez implementado. Allí se identificaron los distintos nodos de hardware y software que intervinieron en la solución, así como la forma en que se organizaron los componentes dentro de la infraestructura disponible. El sistema se desplegó en un servidor virtual de Amazon Web Services (AWS), utilizando una instancia EC2 configurada para ejecutar contenedores Docker. En este entorno se alojaron los servicios del frontend, el backend y la base de datos, los cuales se comunicaron a través de una red interna propia del host.

 El diagrama también reflejó la forma en que los usuarios accedieron a la aplicación, ya fuera desde un navegador web o desde la versión móvil como PWA\cite{pwa-webdev}. En ambos casos, la conexión se estableció a través de Internet hacia la dirección pública del servidor EC2. Gracias a esta representación fue posible visualizar de manera sintética cómo se relacionaron los distintos elementos de la solución y cómo se organizó el entorno de despliegue, lo que sirvió como punto de partida para el análisis detallado de las tecnologías utilizadas en el backend y el frontend.
\cite{inia-uy,aws}
\begin{figure}[htbp]
	\centering
	% Mostrar el diagrama de deployment ocupando la mayor área de la página
	\includegraphics[width=\textwidth,height=0.92\textheight,keepaspectratio]{\detokenize{Deploy UML.png}}
	\caption{Diagrama de \textit{deployment} (UML) que muestra la distribución de nodos y contenedores en la infraestructura de despliegue.}
	\label{fig:deploy}
\end{figure}
\subsection{Tecnologías utilizadas}
\subsubsection{Cliente}
El cliente del sistema fue diseñado como una aplicación web dinámica utilizando Next.js 14 y React 18, tecnologías ampliamente adoptadas en entornos productivos debido a su eficiencia, modularidad y soporte comunitario.
\cite{nextjs}

La estructura se organizó siguiendo las convenciones del App Router de Next.js, lo cual permitió una clara separación entre páginas, componentes y módulos de lógica, promoviendo escalabilidad y mantenibilidad.

Para el desarrollo de la interfaz de usuario se utilizaron Tailwind CSS, Radix UI y Shadcn. Estas tecnologías permitieron crear una experiencia visual coherente y accesible, con componentes reutilizables y mantenibles, alineados con estándares modernos de diseño.
\cite{tailwindcss,radixui,shadcn}

La comunicación entre el frontend y el backend se realizó mediante APIs REST para la mayor parte de las operaciones, mientras que las funcionalidades en tiempo real se implementaron a través de WebSocket, permitiendo la recepción de notificaciones instantáneas y mejorando la experiencia interactiva del usuario.

El aseguramiento de calidad del frontend incluyó la ejecución de pruebas automatizadas mediante Jest y React Testing Library, herramientas ampliamente utilizadas para validar el comportamiento de componentes y flujos de interacción.
\cite{jest,rtl}
\subsubsection{Servidor}
El backend se implementó utilizando Java 21 con Spring Boot 3.5, aplicando una arquitectura en capas que divide la lógica de negocio, los servicios, los controladores web, los repositorios de datos y las configuraciones transversales. Esta estructura facilita el mantenimiento, reduce el acoplamiento y mejora la estabilidad del sistema.
\cite{spring-boot}
\subsubsection{Servicios y API}
La API del sistema fue documentada utilizando OpenAPI/Swagger, permitiendo que tanto desarrolladores como actores externos comprendan las especificaciones de cada endpoint, formatos de datos, tipos de respuestas y códigos de error.
\cite{openapi}

El backend expone servicios REST y un canal de comunicación en tiempo real mediante WebSocket (STOMP), empleado para enviar notificaciones y eventos relevantes sin necesidad de realizar consultas repetitivas al servidor.
\subsubsection{Persistencia y Base de Datos}
La aplicación utiliza PostgreSQL 15, gestionado mediante Spring Data JPA, lo cual permitió automatizar tareas de persistencia y reducir la necesidad de escribir consultas SQL manuales.
\cite{postgresql}

Durante el proceso de pruebas se utilizó H2, una base de datos en memoria que permite ejecutar los tests de forma rápida, aislada y sin requerir infraestructura externa.
\subsubsection{Seguridad}
La seguridad del sistema se abordó mediante:
\begin{itemize}
    \item Spring Security para la protección de endpoints y control de roles.
    \item JWT para la gestión de sesiones sin estado.
    \item Autenticación en dos pasos (2FA) mediante códigos TOTP, fortaleciendo la seguridad del acceso.
\end{itemize}
\cite{spring-security,rfc7519,rfc6238}

    Este conjunto de herramientas permitió cumplir buenas prácticas de seguridad alineadas con estándares modernos.
\subsubsection{Herramientas de Construcción, Testing y DevOps}
El backend fue compilado y gestionado mediante Maven, lo que permitió automatizar tareas de instalación, test y empaquetado.

La calidad del código se reforzó mediante pruebas unitarias utilizando JUnit 5 y Mockito, enfocadas en validar la lógica de negocio, la interacción entre módulos y el correcto comportamiento de los componentes críticos.
\cite{junit5,jacoco}

Finalmente, el uso de Docker y Docker Compose permitió ejecutar el proyecto de forma completa (frontend, backend y base de datos) mediante entornos aislados y reproducibles, asegurando coherencia en las diferentes etapas del desarrollo.
\cite{docker}

La comunicación con la base de datos se realizó a través de JPA/Hibernate, lo que permitió manejar las entidades del dominio de forma eficiente y reducir la carga de escribir consultas SQL manuales.\cite{hibernate} El servicio funcionó dentro de un contenedor Docker, lo que facilitó su despliegue en la instancia EC2 y garantiza un entorno consistente durante todo el ciclo de vida del proyecto.
\subsection{Horas Dedicadas}
A continuación se detallarán los registros de horas por etapa del proyecto. Estos fueron gestionados utilizando un proyecto en Toggl.
\cite{toggl}

% Resumen rápido de horas por etapa
\begin{center}
\begin{tabular}{l r}
\hline
	\textbf{Etapa} & \textbf{Horas} \\
\hline
Etapa 1 & 60 \\
Etapa 2 & 553 \\
Etapa 3 & 86 \\
\hline
\end{tabular}
\end{center}

% Nota: Para que \texttt{\textcolor{red}} funcione, asegúrate de cargar el paquete \texttt{xcolor} en el preámbulo si aún no está incluido.
\subsubsection{Etapa 1: Documentación y Análisis}
\textbf{Período:} Desde el 14 de agosto al 08 de septiembre de 2025.
La primera etapa se enfocó principalmente en recolectar la información pertinente con las clientes y documentar las características inicialmente establecidas para comenzar con el desarrollo adecuado de la aplicación web.

Un total de 60 horas fueron dedicadas a esta fase.
\subsubsection{Etapa 2: Desarrollo}
\textbf{Período:} Desde el 08 de septiembre al 08 de noviembre de 2025.
La etapa de desarrollo se dedicó enteramente al desarrollo completo de la aplicación web, cubriendo los requerimientos solicitados y validando con las clientes.

Un total de 553 horas fueron dedicadas a esta fase.
\subsubsection{Etapa 3: Testing, Documentación Final y Presentación}
\textbf{Período:} Desde el 08 de noviembre al 30 de noviembre de 2025.
La etapa final del proyecto se enfocó en la realización de pruebas funcionales y de usabilidad, así como en la corrección de errores detectados. Además, se completó la documentación técnica y se preparó la presentación final del proyecto.

Un total de 86 horas fueron dedicadas a esta fase.
\section{Gestión de Proyecto}
Para la ejecución de este proyecto no se estructuraron roles específicos para cada integrante del equipo. Si bien cada miembro tenía mayor afinidad con determinadas áreas tecnológicas, se decidió adoptar una dinámica de trabajo colaborativa en la que todos participaron tanto en el desarrollo del frontend como el backend, así como en tareas vinculadas a documentación, testing y despliegue.
\cite{inia-uy}

Para la organización del trabajó se creó un proyecto en Notion, donde se registraron las tareas principales, los responsables y las estimaciones de tiempo para cada tarea. 
\cite{notion}

El principal medio de comunicación en línea fue Google Meet, utilizado para reuniones virtuales, donde a su vez se llevaron a cabo reuniones periódicas con el cliente. Además, se creó un grupo de WhatsApp para coordinar rápidamente temas urgentes, notificaciones internas y actualizaciones cortas entre los integrantes del equipo, lo que permitió una comunicación ágil y continua.
\cite{googlemeet,whatsapp}
\subsection{Entorno de Desarrollo}
El desarrollo del sistema se realizó en un entorno orientado a aplicaciones web modernas, priorizando la reproducibilidad, la trazabilidad y la correcta separación de responsabilidades entre los distintos componentes.
\cite{docker}

Para el desarrollo del software se utilizaron dos entornos de desarrollo integrados (IDE):
\begin{itemize}
    \item Visual Studio Code, empleado principalmente para el frontend y configurado con extensiones específicas para TypeScript, React, Docker y herramientas de control de versiones.
    \cite{vscode}
    \item IntelliJ IDEA, utilizado para el backend en Java, aprovechando su soporte avanzado para Spring Boot, gestión de dependencias con Maven y depuración integrada.
    \cite{intellij}
\end{itemize}
La gestión del código fuente se realizó utilizando Git como sistema de control de versiones y GitHub como plataforma principal de alojamiento, revisión y seguimiento del proyecto. GitHub permitió manejar ramas, realizar revisiones de código, gestionar issues y mantener un flujo de trabajo organizado basado en buenas prácticas de versionado.
\cite{git,github}

Esta combinación permitió a todos los integrantes del equipo trabajar de forma eficiente y con herramientas adecuadas para cada tecnología.

\vspace{1em}
\noindent\textbf{Nota:} Para información detallada sobre la identificación, evaluación y mitigación de riesgos del proyecto, consultar el \textbf{Anexo 3 -- Mitigación de Riesgos} en el documento \texttt{Anexos.pdf}.

\section{Problemas Encontrados}
\subsection{Reestructuración del Módulo de Germinación}
Durante la etapa final del desarrollo se identificó un problema crítico en el módulo de germinación, que obligó a replantear su estructura, lógica interna y funcionamiento general. Este inconveniente surgió a raíz de una discrepancia significativa entre los requisitos originalmente planteados y las necesidades reales del laboratorio\cite{inia-uy,ista2023}.

La dificultad se originó en la documentación inicial proporcionada por el cliente, compuesta por hojas de cálculo con información incompleta, las cuales eran utilizadas por el equipo para realizar la gestión de forma manual. A partir de la referencia el equipo asumió que todas las fechas de conteo eran comunes a todo el análisis de germinación y que las diferentes configuraciones de días de prefrío ingresadas en el mismo análisis no influían en las mismas, entre otras\cite{fao-seed,ista2023}.

Estas suposiciones llevaron a diseñar el módulo bajo una estructura que no reflejaba la complejidad real del proceso.
\subsubsection{Cambio de Requisitos}
Pocos días antes de la finalización de la etapa de desarrollo, el cliente aclaró que el funcionamiento real del análisis de germinación difería de lo inicialmente interpretado. Los requisitos correctos incluían configuraciones múltiples con fechas de conteos independientes, restringidas por los días de prefrío y con validaciones específicas adicionales.

Este cambio presentó un ajuste conceptual profundo respecto a la estructura inicial del módulo, y su corrección requirió una reingeniería completa del módulo, afectando múltiples áreas del sistema. Entre las acciones necesarias se incluyó el rediseño de la estructura de los datos, reescritura de la lógica del sistema y modificación de interfaces.

A pesar de las restricciones de tiempo, el equipo logró implementar correctamente la nueva estructura, la cual resultó ser fundamental para asegurar que el módulo cumpliera adecuadamente con los cambios solicitados por el cliente\cite{inia-uy}.
\section{Testing de la Aplicación Desarrollada}
\subsection{Testing General del Proyecto}
El testing de la aplicación se llevó adelante directamente por quienes trabajaron en el proyecto, ya que no se contó con un equipo de pruebas dedicado. Las verificaciones se fueron realizando a medida que se sumaban nuevas funcionalidades, lo que permitió detectar errores con rapidez y corregirlos antes de avanzar con el resto del desarrollo.
\subsection{Pruebas Manuales Iniciales}
Cada funcionalidad fue probada primero de manera manual, revisando su comportamiento desde la interfaz o enviando solicitudes a la API según correspondiera. Una vez que el sistema estuvo completo, se hizo una revisión general de todos los casos de uso para asegurarse de que los flujos funcionaran de forma coherente y sin generar fallos entre distintas partes de la aplicación.
\subsection{Pruebas en Backend (Spring Boot)}
En el backend, desarrollado con Spring Boot, se implementaron pruebas unitarias e integraciones utilizando JUnit, Mockito y la herramienta de cobertura JaCoCo. Esto permitió verificar la lógica de negocio, los controladores y varios de los servicios internos. La cobertura superó el 80\%, un valor adecuado para tener un buen nivel de confianza en el comportamiento de las áreas más importantes del código.
\cite{spring-boot,junit5,mockito,jacoco}

De todos modos, se entiende que un porcentaje elevado de cobertura no garantiza que el software esté libre de errores. La cobertura solo indica qué líneas fueron ejecutadas durante las pruebas, pero no asegura que todas las variantes, casos límite o comportamientos inesperados hayan sido contemplados. Por esta razón, las pruebas automatizadas se complementaron con verificaciones manuales, exploratorias y de integración, lo que permitió detectar situaciones que no siempre quedan reflejadas en los tests automatizados y, en conjunto, mejorar la calidad general del backend.

\begin{table}[h]
\centering
\begin{tabular}{|l|c|}
\hline
\textbf{Métrica} & \textbf{Resultado} \\
\hline
Coverage total & 95\% \\
\hline
Clases analizadas & 173 \\
\hline
Métodos analizados & 3002 \\
\hline
Tests Ejecutados & 1440 \\
\hline
\end{tabular}
\caption{Métricas de cobertura de pruebas con JaCoCo}
\label{tab:jacoco-metrics}
\end{table}

\subsection{Pruebas en Frontend (Next.js + TypeScript)}
El frontend, construido con Next.js y TypeScript, también incorporó pruebas automatizadas mediante Jest y React Testing Library. Estas pruebas abarcaron componentes, funciones auxiliares y ciertos elementos de la lógica de presentación. Al igual que en el backend, el nivel de cobertura superó el 80\%, lo que ayudó a detectar errores de visualización, manejo de datos y algunos detalles que surgieron durante la integración con la API.
\cite{nextjs,jest,rtl}

\begin{table}[h]
\centering
\begin{tabular}{|l|c|}
\hline
	extbf{Métrica} & \textbf{Resultado} \\
\hline
Coverage total & 86\% \\
\hline
Funciones cubiertas & 82\%-85\% \\
\hline
Tests ejecutados & 1931 \\
\hline
\end{tabular}
\caption{Métricas de cobertura de pruebas con Jest}
\label{tab:jest-metrics}
\end{table}

\subsection{Pruebas Manuales en Web y PWA}
Además de estas pruebas automatizadas, se realizaron pruebas manuales tanto en la versión web como en la PWA\cite{pwa-webdev}. En esta etapa se revisó la navegación completa, el funcionamiento en dispositivos móviles, la instalación de la aplicación y los flujos principales de interacción desde el punto de vista del usuario final.
\subsection{Pruebas \textit{End-to-End} (E2E)}
Por último, se realizaron pruebas \textit{end-to-end} (E2E) para evaluar el funcionamiento del sistema completo, desde la interfaz hasta la base de datos. Este tipo de testing se centró en reproducir escenarios reales de uso, verificando que los distintos componentes interactúan de forma correcta y continua. Las pruebas E2E permitieron validar flujos completos como el inicio de sesión, la carga y consulta de datos, la comunicación con la API y la correcta actualización de la información en la base de datos. Este enfoque fue especialmente útil para detectar errores que no aparecen en pruebas aisladas, como problemas de integración, diferencias en el formato de los datos o comportamientos inesperados al combinar varias funcionalidades en un mismo flujo.
\subsection{Conclusión}
Gracias a esta combinación de pruebas unitarias, integradas, automatizadas, manuales y E2E, se logró obtener un sistema estable, con un buen nivel de calidad y sin fallos críticos al finalizar el desarrollo.

\section{La Solución Desarrollada}
\subsection{Objetivos de la Solución}
% Resumen de los objetivos que cumple la solución implementada.

\subsubsection{Gestión de Lotes de Semillas}
% Descripción del módulo de gestión de lotes.

\subsubsection{Análisis de Calidad}
% Describir procesos y cálculos asociados.

\subsubsection{Pureza Física}
% Detalles sobre la comprobación de pureza física.

\subsubsection{Germinación}
% Módulo y procesos de germinación.

\subsubsection{DOSN (Determinación de Otras Semillas en Número)}
% Explicar este cálculo y su implementación.

\subsubsection{PMS (Peso de Mil Semillas)}
% Cómo se calcula y registra.

\subsubsection{Tetrazolio}
% Proceso y registro de resultados.

\subsubsection{Flujo de Trabajo del Laboratorio}
% Describir el flujo automatizado/manual.

\subsubsection{Reportes y Exportaciones}
% Tipos de reportes y formatos de exportación soportados.

\section{Conclusiones}
El desarrollo del Sistema de Gestión de Laboratorio de Semillas (SGLS) para INIA Uruguay representa un avance clave en la modernización de los procesos de control de calidad. El proyecto surge para sustituir flujos de trabajo manuales basados en planillas independientes, que generaban dispersión de datos, dificultades en la trazabilidad y una alta dependencia del registro humano. Estas limitaciones afectaban la eficiencia diaria del laboratorio y podían comprometer la confiabilidad de los resultados, por lo que avanzar hacia un sistema centralizado y digital se volvió una necesidad estratégica.
Con la implementación del SGLS, se logró integrar en un único entorno todas las etapas del análisis de semillas, desde la recepción de muestras hasta la emisión de informes finales. Este enfoque unificado permitió eliminar la duplicación de datos, mejorar la trazabilidad de cada lote y asegurar un seguimiento claro y preciso a lo largo de todo el proceso. La automatización de cálculos complejos y la validación de información redujo considerablemente los errores humanos, aportando mayor consistencia y confianza en los resultados generados. Esto no solo agiliza el trabajo interno, sino que también fortalece la calidad de los servicios ofrecidos por el laboratorio.
Finalmente, el sistema no solo cumple con los requerimientos actuales del proceso de certificación, sino que establece una base sólida para el crecimiento futuro. Su diseño organizado y adaptable permite incorporar nuevos módulos, integrar equipamiento automatizado y ampliar las capacidades analíticas y de reporte que el laboratorio pueda necesitar más adelante. En conjunto, el SGLS posiciona a INIA como una institución alineada con las demandas modernas del sector agropecuario, demostrando cómo la digitalización puede potenciar la productividad, garantizar la calidad y respaldar la toma de decisiones con datos confiables.
\section{Trabajos a Futuro}
\subsection{Módulo de auditorías y control de calidad}
% Describir alcance y beneficios.

\subsection{Gestión documental avanzada}
% Ideas para mejorar la gestión documental.

\subsection{Automatización de cálculos y validaciones normativas}
% Automatizaciones pendientes o deseadas.

\subsection{Optimización del flujo de trabajo}
% Posibles mejoras en rendimiento o UX.

\subsection{Integración con sistemas externos}
% APIs y conectores deseados.


\section{Referencias}
% Bibliografía gestionada con BibTeX (estilo numérico)
% Crea/actualiza `bibliografia.bib` en el mismo directorio principal del proyecto.
\bibliographystyle{plain}
% Incluir todas las entradas de `bibliografia.bib` aunque no estén citadas
\nocite{*}
\bibliography{bibliografia}


\end{document}
% Documentación del Proyecto Final - Estilo visual APA (numerado)
\documentclass[12pt]{article}
% Idioma y codificación
\usepackage[spanish]{babel}
\usepackage[utf8]{inputenc}
\usepackage[T1]{fontenc}

% Tipografía y espaciado (estilo similar a APA)
\usepackage{mathptmx} % Times-like font (compatible)
\usepackage{setspace}
\doublespacing % APA suele usar interlineado doble

% Márgenes APA: 1 pulgada (2.54 cm)
\usepackage[letterpaper,margin=1in]{geometry}

% Encabezados y formato de secciones
\usepackage{fancyhdr}
\usepackage{titlesec}
\usepackage{graphicx}
\usepackage{enumitem}
\usepackage{hyperref}

% Configuración de encabezado (running head)
\pagestyle{fancy}
\fancyhf{}
\fancyhead[L]{\textit{Título corto del trabajo}}
\fancyhead[R]{\thepage}
\renewcommand{\headrulewidth}{0pt}

% Secciones estilo APA: flush left, bold, no extra vertical space
	itleformat{\section}{\normalfont\bfseries\large}{\thesection.}{0.5em}{}
	itleformat{\subsection}{\normalfont\bfseries}{\thesubsection.}{0.5em}{}

\setcounter{secnumdepth}{4}
\setcounter{tocdepth}{3}

	itle{Documentación del Proyecto Final (APA visual)}
\author{Nombre del Autor}
\date{\today}

\begin{document}
\maketitle
\vspace{1cm}
	ableofcontents
\clearpage

% Secciones principales (numeradas)

\section*{Agradecimientos}
\addcontentsline{toc}{section}{Agradecimientos}

Queremos expresar nuestro más sincero agradecimiento a aquellas personas que nos han acompañado y apoyado a lo largo de nuestra trayectoria académica. En especial, a nuestro docente y tutor Nicolás Escobar, quien ha sido un pilar fundamental en nuestra formación profesional. Sus observaciones, correcciones y sugerencias nos permitieron mejorar la calidad del proyecto y mantener una dirección clara en el proceso de desarrollo.

Agradecemos también a nuestras familias, por su apoyo constante y por facilitar las condiciones necesarias para llevar adelante este trabajo, especialmente durante los períodos de mayor dedicación. Su comprensión y acompañamiento hicieron posible sostener el ritmo de trabajo requerido.

A todas esas personas, nuestro más sincero agradecimiento.


\section*{Resumen}
\phantomsection
\addcontentsline{toc}{section}{Resumen}
El sistema web integral desarrollado para el Instituto Nacional de Investigación Agropecuaria (INIA) surge como una solución tecnológica para modernizar y digitalizar los procesos de análisis de calidad de semillas. El proyecto, realizado como trabajo final de la carrera Tecnólogo en Informática, aborda la necesidad de unificar y estandarizar procedimientos técnicos que anteriormente se gestionaban mediante planillas Excel.

El sistema permite administrar lotes y diversos análisis, como Germinación, Tetrazolio, PMS (Peso de Mil Semillas), DOSN (Determinación de Otras Semillas en Número) y Pureza Física, incorporando validaciones basadas en estándares internacionales, trazabilidad e historial técnico por lote, normalización de catálogos, importación de datos legados y generación de reportes especializados con exportación consolidada a Excel.

La arquitectura se implementó siguiendo un modelo cliente-servidor en capas, con un backend en Java Spring Boot 3.5 expuesto como API REST y reforzado con seguridad por JWT (RFC 7519) y autenticación en dos factores (2FA). Se incorporaron notificaciones en tiempo real mediante WebSockets, paginación \textit{offset} y \textit{limit}, y pruebas automatizadas con JUnit/JaCoCo. El frontend fue desarrollado en Next.js 14 con TypeScript, ofreciendo una PWA (Progressive Web App) optimizada y con un buen rendimiento.

El sistema resultante mejora la trazabilidad de los datos, la consistencia metodológica de los análisis y la eficiencia operativa institucional, estableciendo una base sólida para futuras ampliaciones y para la interoperabilidad con otros sistemas del INIA.

\section{Introducción}
En la última década, la digitalización ha transformado profundamente la gestión de información en sectores productivos y científicos, impulsando nuevos estándares de eficiencia, trazabilidad y precisión en los procesos operativos. En este contexto, el ámbito agropecuario no ha sido la excepción: la demanda de sistemas capaces de centralizar datos, reducir tareas manuales y garantizar la consistencia metodológica se ha vuelto fundamental para sostener la calidad de los análisis y la toma de decisiones institucionales.

Previo a este proyecto, el proceso de análisis de calidad de semillas en el Instituto Nacional de Investigación Agropecuaria (INIA) presentaba desafíos significativos derivados del uso de registros dispersos, planillas Excel independientes y procedimientos manuales con alto riesgo de error\cite{inia-uy}. Los análisis técnicos —como Germinación, Tetrazolio, PMS, DOSN y Pureza Física— se realizaban sobre lotes de semillas cuyos datos se registraban de forma heterogénea, dificultando el seguimiento del historial completo de cada lote\cite{ista2023,fao-seed}. Esta fragmentación generaba esfuerzos adicionales de normalización y verificación, con impacto en los tiempos de respuesta, en la trazabilidad de los resultados y en la comparabilidad histórica de la información técnica. Además, la ausencia de notificaciones oportunas y la falta de catálogos unificados limitaban la coordinación operativa entre equipos.

Frente a esta necesidad, surgió el desarrollo del Sistema Web Integral INIA, una plataforma diseñada para modernizar, centralizar y estandarizar todo el ciclo de vida de la información vinculada a los análisis de calidad de semillas. El proyecto, realizado como trabajo final de la carrera Tecnólogo en Informática, propone una solución que gestiona de manera estructurada los lotes, sus datos técnicos asociados, los contactos involucrados, las validaciones específicas de cada tipo de análisis, el historial completo de resultados y la generación automatizada de reportes. Asimismo, se incorporó la importación de información proveniente de sistemas legados, preservando datos históricos relevantes y facilitando la transición tecnológica.

En conjunto, esta solución tecnológica permitió profesionalizar la gestión de los análisis y de los lotes de semillas, reducir errores operativos, fortalecer la trazabilidad institucional y mejorar sustancialmente la eficiencia en el registro, control y procesamiento de información técnica. El sistema constituye una base sólida para futuras ampliaciones, nuevos módulos de interoperabilidad y una evolución continua hacia un ecosistema digital integral para el INIA.

\section{Objetivos Planteados y Resultados Esperados}
El principal objetivo del proyecto fue analizar, diseñar y desarrollar un sistema integral para la gestión de análisis de semillas realizados por el Instituto Nacional de Investigación Agropecuaria, orientado a la optimización del proceso actual de registro, edición, consulta y trazabilidad de las muestras (lotes) ingresadas para su análisis. Este sistema busca centralizar y modernizar la operativa ya existente, brindando tecnologías que faciliten la agilizar tareas, reducir errores y mejorar la disponibilidad de información para todos los posibles usuarios.
Durante el proceso de construcción del sistema se evaluaron diferentes alternativas tecnológicas para su implementación, optando por Java 21 junto con Springboot 3.5 y Spring Data JPA para el backend REST. Mientras que en el frontend se consideró React con Tailwind CSS 4.1. Asimismo se buscó la integración de módulos complementarios, como notificaciones, reportes y la exportación e importación de archivos formato xlsx.

\subsection{Objetivos específicos}
\begin{itemize}
  \item Un sistema centralizado y estable para gestionar todo el ciclo de análisis.
  \item Mayor trazabilidad y transparencia en el seguimiento de lotes.
  \item Reducción de errores mediante automatización y validaciones.
  \item Disminución del tiempo dedicado a tareas administrativas.
  \item Operativa más moderna, simple e intuitiva.
\end{itemize}

\section{Estado del Arte}
En los últimos años, la transformación digital se ha vuelto una realidad en prácticamente todas las organizaciones, tanto públicas como privadas. Este proceso busca mejorar la forma en que se gestionan los datos y se llevan adelante las tareas diarias. En el sector agropecuario, especialmente, la digitalización ha demostrado ser clave para lograr una mayor trazabilidad, precisión en los registros y eficiencia operativa. Por este motivo, cada vez más instituciones están dejando atrás los documentos en papel, las planillas de Excel y los procedimientos dispersos, y están migrando hacia plataformas web que centralizan y organizan la información de forma más segura y confiable. Esta tendencia está ampliamente respaldada por estudios académicos y por la práctica común de centros de investigación y laboratorios en todo el mundo.

\cite{inia-uy,ista2023,fao-seed}

El proyecto desarrollado para el INIA se inscribe en esta línea de evolución tecnológica. La institución necesita una herramienta digital que permita gestionar sus análisis de forma integral: registrar y consultar muestras (lotes), administrar configuraciones, controlar distintos roles de usuario, generar reportes, exportar información y también incorporar datos históricos provenientes de sistemas anteriores. Este capítulo presenta el marco conceptual, tecnológico y comparativo que justifica la solución propuesta y contextualiza su diseño dentro de las tendencias actuales de la industria del software.

\cite{inia-uy}
\subsection{Marco Teórico}
En estos últimos años las organizaciones adoptaron hojas de cálculo como herramienta base para almacenar información gracias a su accesibilidad, bajo costo, flexibilidad y facilidad de uso. Sin embargo, a medida que los procesos se vuelven más complejos y los volúmenes de datos crecen, estos mecanismos se vuelven insuficientes y arriesgados para labores críticas. Las limitaciones más comunes incluyen:
\begin{itemize}
    \item \textbf{Escalabilidad restringida:} Las hojas de cálculo no están diseñadas para manejar grandes volúmenes de datos ni crecer de forma sostenible. A medida que aumentan los registros, las operaciones se vuelven lentas, propensas a fallos y difíciles de mantener.
    \item \textbf{Colaboración limitada:} La funcionalidad de edición simultánea que es ofrecida por la mayoría de las aplicaciones de gestión de hojas de cálculo es vulnerable a conflictos, sobreescrituras y pérdida de información.
    \item \textbf{Falta de trazabilidad y mecanismos de auditoría:} Resulta difícil rastrear cambios, identificar responsables y asegurar integridad de datos.
    \item \textbf{Integración deficiente con sistemas externos:} La conexión con APIs, bases de datos, servicios externos u otros sistemas institucionales resulta limitada.
\end{itemize}
\subsection{La Transición Hacia Sistemas Web}
Las planillas suelen funcionar bien en etapas iniciales, pero a medida que crecen los volúmenes de información o los procesos se vuelven más complejos, es común que se busquen alternativas web que permitan centralizar la información y manejarla de manera más eficiente. Los sistemas modernos ofrecen:
\begin{itemize}
    \item Acceso multiplataforma.
    \item Escalabilidad horizontal mediante arquitecturas distribuidas.
    \item Integración nativa a través de APIs REST.
    \item Validaciones automáticas a nivel de negocio.
    \item Auditorías completas de operaciones.
    \item Estandarización de procesos y flujos de trabajo.
\end{itemize}
\cite{ista2023,fao-seed}
Este enfoque tecnológico es el más utilizado en laboratorios, centros de investigación y organizaciones científicas para gestionar muestras, controles de calidad, análisis y trazabilidad de procesos experimentales.
\subsection{Tecnologías}
El proyecto adopta tecnologías modernas, robustas y ampliamente utilizadas en la industria. La selección busca equilibrio entre estabilidad, madurez, facilidad de mantenimiento y alineación con estándares profesionales.
\cite{spring-boot,nextjs}
\subsubsection{Frontend: React + TypeScript}
React es uno de los frameworks más utilizados globalmente para construir interfaces de usuario debido a:
\begin{itemize}
    \item Su modelo de componentes reutilizables
    \item Su alto rendimiento mediante virtual DOM
    \item Un ecosistema amplio de librerías
    \item Facilidad para construir SPA (Single Page Applications)
\end{itemize}
El uso de TypeScript aporta tipado estático, reduce errores y mejora la mantenibilidad del código.
\subsubsection{Backend: Java + Spring Boot}
Spring Boot es estándar en el desarrollo empresarial gracias a:
\begin{itemize}
    \item Integración nativa con Spring Security
    \item Soporte simplificado para APIs REST
    \item Inyección de dependencias y modularidad
    \item Comunidad madura y documentación extensa
\end{itemize}
Java 21, como versión LTS, asegura estabilidad a largo plazo.
\subsubsection{Base de Datos: PostgreSQL}
Elegida por ser:
\begin{itemize}
    \item Open source
    \item Altamente confiable
    \item Compatible con operaciones complejas
    \item Escalable y sólida para manejo de datos institucionales
\end{itemize}
\subsubsection{Infraestructura y Arquitectura}
\begin{itemize}
    \item Modelo cliente-servidor (frontend Next.js – backend Spring Boot)
    \item Arquitectura en tres capas (presentación – servicios – datos)
    \item Uso de Docker para portabilidad y despliegue
    \item Estructura modular por dominios (análisis, seguridad, notificaciones, usuarios)
    \item Comunicación vía REST y WebSocket (notificaciones en tiempo real) \cite{rfc6455,rfc7519}
\end{itemize}
\subsection{Soluciones Similares}
En el mercado existen herramientas orientadas a gestionar información estructurada, colaborar en equipos y reemplazar flujos basados en planillas. Aunque ninguna se adapta exactamente a los procesos complejos del INIA, sirven como referencia sobre cómo la industria resuelve problemas similares.
\subsubsection{Airtable}
Airtable combina conceptos de base de datos con la interfaz amigable de una hoja de cálculo. Es una plataforma low-code orientada a pequeños proyectos y equipos que necesitan digitalizar procesos sin desarrollar software propio.

Ventajas:
\begin{itemize}
    \item Interfaz simple y accesible
    \item Colaboración en tiempo real
    \item APIs integradas
    \item Automatizaciones básicas
\end{itemize}
Limitaciones:
\begin{itemize}
    \item No escala para flujos complejos
    \item Restricciones para reglas de negocio avanzadas
    \item Dependencia de licencias externas
\end{itemize}
\cite{airtable}
\begin{figure}[htbp]
    \centering
    \includegraphics[width=0.8\textwidth,height=0.6\textheight,keepaspectratio]{Airtable.png}
    \caption{Ejemplo: vista de contenido en Airtable (tabla/registro).}
    \label{fig:airtable}
\end{figure}
\subsubsection{Smartsheet}
Smartsheet ofrece una experiencia similar a Excel, pero con mayor control, trazabilidad y herramientas para flujos de trabajo.

Ventajas:
\begin{itemize}
    \item Gestión de proyectos y procesos
    \item Reportes avanzados
    \item Automatizaciones integradas
\end{itemize}
Limitaciones:
\begin{itemize}
    \item Alto costo según uso
    \item Menor flexibilidad frente a un desarrollo a medida
    \item Dependencia del ecosistema propietario
\end{itemize}
\cite{smartsheet}
\begin{figure}[htbp]
    \centering
    \includegraphics[width=0.8\textwidth,height=0.6\textheight,keepaspectratio]{Smartsheet.png}
    \caption{Ejemplo: interfaz de Smartsheet mostrando una vista tipo hoja de cálculo y reportes.}
    \label{fig:smartsheet}
\end{figure}
\subsubsection{Odoo}
Odoo es un ERP (Planificador de Recursos Empresariales) modular que integra distintos dominios empresariales.

Ventajas:
\begin{itemize}
    \item Gran variedad de módulos
    \item Comunidad activa
    \item Escalabilidad para múltiples áreas
\end{itemize}
Limitaciones:
\begin{itemize}
    \item Instalación y configuración complejas
    \item Sobrecarga funcional para proyectos específicos
    \item Dificultad de adaptación a procesos científicos y de laboratorio
\end{itemize}
\cite{odoo}
\begin{figure}[htbp]
    \centering
    \includegraphics[width=0.8\textwidth,height=0.6\textheight,keepaspectratio]{Odoo.png}
    \caption{Ejemplo: módulo de Odoo con vistas integradas de gestión de recursos.}
    \label{fig:odoo}
\end{figure}
\subsection{Conclusión del análisis comparativo}
Si bien las herramientas ofrecen funcionalidades útiles, no cumplen de forma precisa con los requerimientos del INIA, especialmente en lo relativo a:
\begin{itemize}
    \item Procesos de laboratorio
    \item Gestión de análisis y múltiples roles
    \item Validaciones específicas
    \item Importación de datos legados
    \item Adaptación a flujos técnicos
\end{itemize}
Por ello, un sistema a medida es la opción más adecuada para atender las necesidades reales y garantizar la adaptación total a los procesos internos y ofrecer la flexibilidad necesaria para futuros cambios o expansiones.
\cite{inia-uy,ista2023}
\section{Análisis del Problema}
Para poder desarrollar el sistema web del INIA fue necesario entender a fondo cuáles eran los problemas reales en el trabajo diario del laboratorio. Durante mucho tiempo los análisis se registraban en planillas haciendo difícil mantener un control claro y asegurar la uniformidad de la información. Este diagnóstico inicial fue clave para diseñar una solución que realmente mejorase la forma en que los análisis se gestionan y resultase útil para el cliente.
A partir de este análisis, se estudiaron cuidadosamente los procesos actuales, se conversó con los usuarios y se documentaron sus necesidades. Esto permitió definir cómo debía funcionar el sistema, qué información era importante, quiénes interactúan con él y qué herramientas debían construirse para hacer su trabajo más sencillo y seguro. Con esta información se estableció el alcance del proyecto y se identificaron los puntos críticos a resolver en cuanto a diseño y desarrollo.
\subsection{Vista del Modelo de Dominio}
El modelo de dominio reúne todas las entidades del sistema y muestra cómo se relacionan todas entre sí. Tener esta vista clara fue fundamental para ordenar las ideas, entender el flujo de la información y asegurarse de que el sistema reflejase fielmente la forma en que se trabaja en el laboratorio.
\begin{figure}[htbp]
    \centering
    \includegraphics[width=0.8\textwidth,height=0.6\textheight,keepaspectratio]{\detokenize{Diagrama de Dominio.png}}
    \caption{Modelo de dominio: entidades y relaciones principales.}
    \label{fig:modelo_dominio}
\end{figure}
\subsection{Definición de Casos de Uso}
Los casos de uso ayudan a describir qué puede hacer cada usuario y cómo se relacionan las acciones con las funcionalidades disponibles en el sistema. A partir de ellos se identificaron tareas clave como crear y gestionar lotes y análisis, administrar catálogos, generar reportes y cargar datos históricos.
\subsection{Actores}
Los actores son los elementos externos al sistema, ya sean usuarios o sistemas, que interactúan con la plataforma para llevar a cabo determinadas tareas. En este proyecto se definieron tres actores centrales:

\textbf{Administrador:}
\begin{itemize}
    \item Administra catálogos, lista de contactos y aprobación de usuarios.
    \item Gestiona análisis y lotes.
    \item Importa datos históricos.
    \item Otorga y revoca permisos.
    \item Exporta reportes en formato xlsx.
    \item Supervisa y valida las acciones realizadas por los Analistas.
    \item Observa reportes.
    \item Tiene acceso total a todos los módulos del sistema.
\end{itemize}
\textbf{Analista:}
\begin{itemize}
    \item Registra y edita lotes y análisis.
    \item Carga y modifica información técnica.
    \item Requiere la aprobación del Administrador para confirmar los resultados de un análisis.
    \item Exporta reportes en formato xlsx.
    \item Observa reportes.
\end{itemize}
\textbf{Observador:}
\begin{itemize}
    \item Solo puede visualizar información.
    \item Accede a lotes, análisis, resultados y reportes.
    \item No posee permisos para modificar datos.
\end{itemize}
\subsection{Diagrama de Casos de Uso}
El siguiente diagrama muestra los principales casos de uso del sistema:
\begin{figure}[htbp]
    \centering
    \includegraphics[width=0.8\textwidth,height=0.6\textheight,keepaspectratio]{\detokenize{Diagrama Casos de Uso.png}}
    \caption{Diagrama de casos de uso: principales interacciones entre actores y el sistema.}
    \label{fig:casos_uso}
\end{figure}
\subsection{Vista del Modelo de Diseño}
El diseño propuesto para la arquitectura del sistema busca mantener un equilibrio claro entre simplicidad, organización y capacidad de crecimiento. La estructura en capas responde a la necesidad de separar responsabilidades y asegurar que cada parte del sistema pueda evolucionar sin generar impacto innecesario en las demás. Este enfoque no solo mejora la mantenibilidad, sino que también facilita la incorporación de nuevas funcionalidades en el futuro. 
La organización del sistema se basa en una arquitectura por capas que incluye la capa de Cliente, donde se ubica la aplicación web, incluyendo sus capacidades de PWA, y la capa de Presentación, encargada de gestionar la interfaz y la comunicación inicial con el backend. A esto se suma la capa de Seguridad, que centraliza los mecanismos de autenticación, autorización y protección de datos. Por su parte, la capa de Aplicación, implementada con Spring Boot, contiene toda la lógica del negocio y se encarga del flujo de información hacia la capa de Datos, responsable del acceso y persistencia.
En esta arquitectura, los servicios adicionales que utiliza el sistema, como el envío de correos o la verificación en dos pasos, se integran directamente dentro de la capa de Aplicación. Esto evita complejidades innecesarias en la representación del diseño y mantiene el diagrama coherente y simple, sin dejar de reflejar el funcionamiento real del sistema.
\begin{figure}[htbp]
    \centering
    \includegraphics[width=0.8\textwidth,height=0.6\textheight,keepaspectratio]{\detokenize{Diagrama de Arquitectura.png}}
    \caption{Diagrama de arquitectura: capas y componentes principales del sistema.}
    \label{fig:arquitectura}
\end{figure}
\subsection{Descripción de la Arquitectura del Sistema}
La arquitectura propuesta organiza el sistema en capas bien definidas, lo que permite mantener una estructura ordenada, escalable y fácil de mantener. Cada capa cumple un rol específico dentro del flujo general de la aplicación.
\subsubsection{Capa de Cliente (SPA/PWA)}
El frontend funciona como una aplicación web moderna basada en Next.js y diseñada como SPA con capacidades PWA. Desde el navegador, los usuarios interactúan mediante una interfaz rápida, responsiva y adaptable a distintos dispositivos. La aplicación soporta instalación como app y toda la comunicación con el backend se realiza a través de REST.
\subsubsection{Capa de Seguridad}
Incluye los mecanismos que protegen el acceso al sistema. Se utiliza autenticación basada en JWT almacenado en cookies seguras y un sistema de doble factor (2FA) mediante códigos TOTP compatibles con Google Authenticator. La autorización se gestiona con Spring Security mediante roles como ADMIN, ANALISTA y OBSERVADOR. El frontend también aporta seguridad con middleware que controla el acceso a rutas protegidas. Además, existe un sistema de notificaciones por correo para avisos de seguridad y recuperación de cuentas.
\subsubsection{Capa de Presentación (Frontend)}
Esta capa engloba la lógica de presentación, los componentes visuales, la validación de formularios y el manejo de estado. Se utilizan herramientas como React Query, React Hook Form, Zod, Radix UI, shadcn/ui y Tailwind. El frontend organiza sus rutas y funcionalidades en módulos bien definidos: autenticación, administración, análisis, reportes, perfil, notificaciones, etc. También ofrece integración con WebSockets para recibir actualizaciones y notificaciones automáticamente e incluye funcionalidades como dashboards interactivos, listados, formularios complejos e instalación como PWA.
\subsubsection{Capa de Aplicación (Backend)}
El backend está construido con Spring Boot siguiendo el patrón MVC y estructurado en Controllers, Services y Repositories. Los Controllers exponen las APIs, los Services contienen la lógica de negocio y los Repositories gestionan el acceso a la base de datos. Aquí se manejan análisis de semillas, validaciones, reportes, importación de datos históricos, seguridad, notificaciones y todo el flujo del sistema. También se implementa un canal WebSocket con STOMP para notificaciones en tiempo real, manejo de transacciones y un sistema global de manejo de excepciones.
\subsubsection{Capa de Datos}
La persistencia está implementada con Spring Data JPA e Hibernate, usando PostgreSQL como base de datos. La base está diseñada con relaciones bien definidas e integridad referencial.
\subsection{Descomposición en subsistemas}
El sistema se organiza en cinco subsistemas principales que trabajan en conjunto para ofrecer una plataforma robusta, segura y orientada a la experiencia del usuario. Cada uno cumple un rol específico dentro de la solución, y en conjunto conforman una arquitectura coherente y fácil de mantener. A continuación, se detalla cada subsistema y los elementos que lo componen.
\subsubsection{Subsistema de Cliente}
Representa lo que usa el usuario en su navegador o como PWA instalada.
\begin{itemize}
    \item Renderiza la interfaz y maneja la experiencia visual e interacción.
    \item Funciona incluso con conexión inestable
    \item Administra cookies de sesión y comunicación con el backend vía HTTPS y WebSockets.
    \item Se encarga de notificaciones push y la instalación como app en dispositivos.
\end{itemize}
\subsubsection{Subsistema de Seguridad}
\begin{itemize}
    \item Gestiona la autenticación con JWT y Cookies HttpOnly.
    \item Implementa 2FA con TOTP y manejo de dispositivos confiables.
    \item Controla los permisos según roles: ADMIN, ANALISTA, OBSERVADOR.
    \item Administra contraseñas, recuperación de acceso y validaciones.
    \item Incluye middleware que protege rutas y verifica sesiones en el frontend.
    \item Genera alertas por eventos críticos de seguridad (cambios de clave).
\end{itemize}
\subsubsection{Subsistema de Presentación (Frontend)}
\begin{itemize}
    \item SPA/PWA construida con componentes reutilizables.
    \item Renderiza pantallas, formularios, tablas, flujos y navegación interna.
    \item Maneja el estado global de la aplicación.
    \item Canaliza toda la comunicación hacia el backend mediante servicios API.
    \item Recibe actualizaciones en tiempo real mediante WebSockets.
    \item Aplica reglas de visibilidad según rol del usuario.
\end{itemize}
\subsubsection{Subsistema de Aplicación (Backend)}
\begin{itemize}
    \item Contiene la lógica de negocio principal del sistema.
    \item Expone endpoints REST desde los controllers.
    \item Ejecuta reglas de negocio, validaciones e integraciones internas.
    \item Administra transacciones y operaciones complejas de múltiples pasos.
    \item Gestiona módulos como catálogos, reportes, usuarios, análisis, lotes, etc.
    \item Envía notificaciones en tiempo real al frontend.
    \item Devuelve respuestas seguras, consistentes y auditables.
\end{itemize}
\subsubsection{Subsistema de Datos}
\begin{itemize}
    \item Gestiona el acceso a la base de datos PostgreSQL.
    \item Define las entidades del dominio y su estructura.
    \item Implementa repositorios JPA para consultas y persistencia.
    \item Garantiza integridad, consistencia y trazabilidad de la información.
    \item Registra auditoría automática de creación y modificación de registros.
\end{itemize}
\section{Implementación}
\subsection{Diagrama de \textit{Deployment} de UML}
El diagrama de \textit{Deployment} permite mostrar de manera clara cómo quedó distribuida la arquitectura física del sistema una vez implementado. Allí se identificaron los distintos nodos de hardware y software que intervinieron en la solución, así como la forma en que se organizaron los componentes dentro de la infraestructura disponible. El sistema se desplegó en un servidor virtual de Amazon Web Services (AWS), utilizando una instancia EC2 configurada para ejecutar contenedores Docker. En este entorno se alojaron los servicios del frontend, el backend y la base de datos, los cuales se comunicaron a través de una red interna propia del host.

 El diagrama también reflejó la forma en que los usuarios accedieron a la aplicación, ya fuera desde un navegador web o desde la versión móvil como PWA\cite{pwa-webdev}. En ambos casos, la conexión se estableció a través de Internet hacia la dirección pública del servidor EC2. Gracias a esta representación fue posible visualizar de manera sintética cómo se relacionaron los distintos elementos de la solución y cómo se organizó el entorno de despliegue, lo que sirvió como punto de partida para el análisis detallado de las tecnologías utilizadas en el backend y el frontend.
\cite{inia-uy,aws}
\begin{figure}[htbp]
	\centering
	% Mostrar el diagrama de deployment ocupando la mayor área de la página
	\includegraphics[width=\textwidth,height=0.92\textheight,keepaspectratio]{\detokenize{Deploy UML.png}}
	\caption{Diagrama de \textit{deployment} (UML) que muestra la distribución de nodos y contenedores en la infraestructura de despliegue.}
	\label{fig:deploy}
\end{figure}
\subsection{Tecnologías utilizadas}
\subsubsection{Cliente}
El cliente del sistema fue diseñado como una aplicación web dinámica utilizando Next.js 14 y React 18, tecnologías ampliamente adoptadas en entornos productivos debido a su eficiencia, modularidad y soporte comunitario.
\cite{nextjs}

La estructura se organizó siguiendo las convenciones del App Router de Next.js, lo cual permitió una clara separación entre páginas, componentes y módulos de lógica, promoviendo escalabilidad y mantenibilidad.

Para el desarrollo de la interfaz de usuario se utilizaron Tailwind CSS, Radix UI y Shadcn. Estas tecnologías permitieron crear una experiencia visual coherente y accesible, con componentes reutilizables y mantenibles, alineados con estándares modernos de diseño.
\cite{tailwindcss,radixui,shadcn}

La comunicación entre el frontend y el backend se realizó mediante APIs REST para la mayor parte de las operaciones, mientras que las funcionalidades en tiempo real se implementaron a través de WebSocket, permitiendo la recepción de notificaciones instantáneas y mejorando la experiencia interactiva del usuario.

El aseguramiento de calidad del frontend incluyó la ejecución de pruebas automatizadas mediante Jest y React Testing Library, herramientas ampliamente utilizadas para validar el comportamiento de componentes y flujos de interacción.
\cite{jest,rtl}
\subsubsection{Servidor}
El backend se implementó utilizando Java 21 con Spring Boot 3.5, aplicando una arquitectura en capas que divide la lógica de negocio, los servicios, los controladores web, los repositorios de datos y las configuraciones transversales. Esta estructura facilita el mantenimiento, reduce el acoplamiento y mejora la estabilidad del sistema.
\cite{spring-boot}
\subsubsection{Servicios y API}
La API del sistema fue documentada utilizando OpenAPI/Swagger, permitiendo que tanto desarrolladores como actores externos comprendan las especificaciones de cada endpoint, formatos de datos, tipos de respuestas y códigos de error.
\cite{openapi}

El backend expone servicios REST y un canal de comunicación en tiempo real mediante WebSocket (STOMP), empleado para enviar notificaciones y eventos relevantes sin necesidad de realizar consultas repetitivas al servidor.
\subsubsection{Persistencia y Base de Datos}
La aplicación utiliza PostgreSQL 15, gestionado mediante Spring Data JPA, lo cual permitió automatizar tareas de persistencia y reducir la necesidad de escribir consultas SQL manuales.
\cite{postgresql}

Durante el proceso de pruebas se utilizó H2, una base de datos en memoria que permite ejecutar los tests de forma rápida, aislada y sin requerir infraestructura externa.
\subsubsection{Seguridad}
La seguridad del sistema se abordó mediante:
\begin{itemize}
    \item Spring Security para la protección de endpoints y control de roles.
    \item JWT para la gestión de sesiones sin estado.
    \item Autenticación en dos pasos (2FA) mediante códigos TOTP, fortaleciendo la seguridad del acceso.
\end{itemize}
\cite{spring-security,rfc7519,rfc6238}

    Este conjunto de herramientas permitió cumplir buenas prácticas de seguridad alineadas con estándares modernos.
\subsubsection{Herramientas de Construcción, Testing y DevOps}
El backend fue compilado y gestionado mediante Maven, lo que permitió automatizar tareas de instalación, test y empaquetado.

La calidad del código se reforzó mediante pruebas unitarias utilizando JUnit 5 y Mockito, enfocadas en validar la lógica de negocio, la interacción entre módulos y el correcto comportamiento de los componentes críticos.
\cite{junit5,jacoco}

Finalmente, el uso de Docker y Docker Compose permitió ejecutar el proyecto de forma completa (frontend, backend y base de datos) mediante entornos aislados y reproducibles, asegurando coherencia en las diferentes etapas del desarrollo.
\cite{docker}

La comunicación con la base de datos se realizó a través de JPA/Hibernate, lo que permitió manejar las entidades del dominio de forma eficiente y reducir la carga de escribir consultas SQL manuales.\cite{hibernate} El servicio funcionó dentro de un contenedor Docker, lo que facilitó su despliegue en la instancia EC2 y garantiza un entorno consistente durante todo el ciclo de vida del proyecto.
\subsection{Horas Dedicadas}
A continuación se detallarán los registros de horas por etapa del proyecto. Estos fueron gestionados utilizando un proyecto en Toggl.
\cite{toggl}

% Resumen rápido de horas por etapa
\begin{center}
\begin{tabular}{l r}
\hline
	\textbf{Etapa} & \textbf{Horas} \\
\hline
Etapa 1 & 60 \\
Etapa 2 & 553 \\
Etapa 3 & 86 \\
\hline
\end{tabular}
\end{center}

% Nota: Para que \texttt{\textcolor{red}} funcione, asegúrate de cargar el paquete \texttt{xcolor} en el preámbulo si aún no está incluido.
\subsubsection{Etapa 1: Documentación y Análisis}
\textbf{Período:} Desde el 14 de agosto al 08 de septiembre de 2025.
La primera etapa se enfocó principalmente en recolectar la información pertinente con las clientes y documentar las características inicialmente establecidas para comenzar con el desarrollo adecuado de la aplicación web.

Un total de 60 horas fueron dedicadas a esta fase.
\subsubsection{Etapa 2: Desarrollo}
\textbf{Período:} Desde el 08 de septiembre al 08 de noviembre de 2025.
La etapa de desarrollo se dedicó enteramente al desarrollo completo de la aplicación web, cubriendo los requerimientos solicitados y validando con las clientes.

Un total de 553 horas fueron dedicadas a esta fase.
\subsubsection{Etapa 3: Testing, Documentación Final y Presentación}
\textbf{Período:} Desde el 08 de noviembre al 30 de noviembre de 2025.
La etapa final del proyecto se enfocó en la realización de pruebas funcionales y de usabilidad, así como en la corrección de errores detectados. Además, se completó la documentación técnica y se preparó la presentación final del proyecto.

Un total de 86 horas fueron dedicadas a esta fase.
\section{Gestión de Proyecto}
Para la ejecución de este proyecto no se estructuraron roles específicos para cada integrante del equipo. Si bien cada miembro tenía mayor afinidad con determinadas áreas tecnológicas, se decidió adoptar una dinámica de trabajo colaborativa en la que todos participaron tanto en el desarrollo del frontend como el backend, así como en tareas vinculadas a documentación, testing y despliegue.
\cite{inia-uy}

Para la organización del trabajó se creó un proyecto en Notion, donde se registraron las tareas principales, los responsables y las estimaciones de tiempo para cada tarea. 
\cite{notion}

El principal medio de comunicación en línea fue Google Meet, utilizado para reuniones virtuales, donde a su vez se llevaron a cabo reuniones periódicas con el cliente. Además, se creó un grupo de WhatsApp para coordinar rápidamente temas urgentes, notificaciones internas y actualizaciones cortas entre los integrantes del equipo, lo que permitió una comunicación ágil y continua.
\cite{googlemeet,whatsapp}
\subsection{Entorno de Desarrollo}
El desarrollo del sistema se realizó en un entorno orientado a aplicaciones web modernas, priorizando la reproducibilidad, la trazabilidad y la correcta separación de responsabilidades entre los distintos componentes.
\cite{docker}

Para el desarrollo del software se utilizaron dos entornos de desarrollo integrados (IDE):
\begin{itemize}
    \item Visual Studio Code, empleado principalmente para el frontend y configurado con extensiones específicas para TypeScript, React, Docker y herramientas de control de versiones.
    \cite{vscode}
    \item IntelliJ IDEA, utilizado para el backend en Java, aprovechando su soporte avanzado para Spring Boot, gestión de dependencias con Maven y depuración integrada.
    \cite{intellij}
\end{itemize}
La gestión del código fuente se realizó utilizando Git como sistema de control de versiones y GitHub como plataforma principal de alojamiento, revisión y seguimiento del proyecto. GitHub permitió manejar ramas, realizar revisiones de código, gestionar issues y mantener un flujo de trabajo organizado basado en buenas prácticas de versionado.
\cite{git,github}

Esta combinación permitió a todos los integrantes del equipo trabajar de forma eficiente y con herramientas adecuadas para cada tecnología.

\vspace{1em}
\noindent\textbf{Nota:} Para información detallada sobre la identificación, evaluación y mitigación de riesgos del proyecto, consultar el \textbf{Anexo 3 -- Mitigación de Riesgos} en el documento \texttt{Anexos.pdf}.

\section{Problemas Encontrados}
\subsection{Reestructuración del Módulo de Germinación}
Durante la etapa final del desarrollo se identificó un problema crítico en el módulo de germinación, que obligó a replantear su estructura, lógica interna y funcionamiento general. Este inconveniente surgió a raíz de una discrepancia significativa entre los requisitos originalmente planteados y las necesidades reales del laboratorio\cite{inia-uy,ista2023}.

La dificultad se originó en la documentación inicial proporcionada por el cliente, compuesta por hojas de cálculo con información incompleta, las cuales eran utilizadas por el equipo para realizar la gestión de forma manual. A partir de la referencia el equipo asumió que todas las fechas de conteo eran comunes a todo el análisis de germinación y que las diferentes configuraciones de días de prefrío ingresadas en el mismo análisis no influían en las mismas, entre otras\cite{fao-seed,ista2023}.

Estas suposiciones llevaron a diseñar el módulo bajo una estructura que no reflejaba la complejidad real del proceso.
\subsubsection{Cambio de Requisitos}
Pocos días antes de la finalización de la etapa de desarrollo, el cliente aclaró que el funcionamiento real del análisis de germinación difería de lo inicialmente interpretado. Los requisitos correctos incluían configuraciones múltiples con fechas de conteos independientes, restringidas por los días de prefrío y con validaciones específicas adicionales.

Este cambio presentó un ajuste conceptual profundo respecto a la estructura inicial del módulo, y su corrección requirió una reingeniería completa del módulo, afectando múltiples áreas del sistema. Entre las acciones necesarias se incluyó el rediseño de la estructura de los datos, reescritura de la lógica del sistema y modificación de interfaces.

A pesar de las restricciones de tiempo, el equipo logró implementar correctamente la nueva estructura, la cual resultó ser fundamental para asegurar que el módulo cumpliera adecuadamente con los cambios solicitados por el cliente\cite{inia-uy}.
\section{Testing de la Aplicación Desarrollada}
\subsection{Testing General del Proyecto}
El testing de la aplicación se llevó adelante directamente por quienes trabajaron en el proyecto, ya que no se contó con un equipo de pruebas dedicado. Las verificaciones se fueron realizando a medida que se sumaban nuevas funcionalidades, lo que permitió detectar errores con rapidez y corregirlos antes de avanzar con el resto del desarrollo.
\subsection{Pruebas Manuales Iniciales}
Cada funcionalidad fue probada primero de manera manual, revisando su comportamiento desde la interfaz o enviando solicitudes a la API según correspondiera. Una vez que el sistema estuvo completo, se hizo una revisión general de todos los casos de uso para asegurarse de que los flujos funcionaran de forma coherente y sin generar fallos entre distintas partes de la aplicación.
\subsection{Pruebas en Backend (Spring Boot)}
En el backend, desarrollado con Spring Boot, se implementaron pruebas unitarias e integraciones utilizando JUnit, Mockito y la herramienta de cobertura JaCoCo. Esto permitió verificar la lógica de negocio, los controladores y varios de los servicios internos. La cobertura superó el 80\%, un valor adecuado para tener un buen nivel de confianza en el comportamiento de las áreas más importantes del código.
\cite{spring-boot,junit5,mockito,jacoco}

De todos modos, se entiende que un porcentaje elevado de cobertura no garantiza que el software esté libre de errores. La cobertura solo indica qué líneas fueron ejecutadas durante las pruebas, pero no asegura que todas las variantes, casos límite o comportamientos inesperados hayan sido contemplados. Por esta razón, las pruebas automatizadas se complementaron con verificaciones manuales, exploratorias y de integración, lo que permitió detectar situaciones que no siempre quedan reflejadas en los tests automatizados y, en conjunto, mejorar la calidad general del backend.

\begin{table}[h]
\centering
\begin{tabular}{|l|c|}
\hline
\textbf{Métrica} & \textbf{Resultado} \\
\hline
Coverage total & 95\% \\
\hline
Clases analizadas & 173 \\
\hline
Métodos analizados & 3002 \\
\hline
Tests Ejecutados & 1440 \\
\hline
\end{tabular}
\caption{Métricas de cobertura de pruebas con JaCoCo}
\label{tab:jacoco-metrics}
\end{table}

\subsection{Pruebas en Frontend (Next.js + TypeScript)}
El frontend, construido con Next.js y TypeScript, también incorporó pruebas automatizadas mediante Jest y React Testing Library. Estas pruebas abarcaron componentes, funciones auxiliares y ciertos elementos de la lógica de presentación. Al igual que en el backend, el nivel de cobertura superó el 80\%, lo que ayudó a detectar errores de visualización, manejo de datos y algunos detalles que surgieron durante la integración con la API.
\cite{nextjs,jest,rtl}

\begin{table}[h]
\centering
\begin{tabular}{|l|c|}
\hline
	extbf{Métrica} & \textbf{Resultado} \\
\hline
Coverage total & 86\% \\
\hline
Funciones cubiertas & 82\%-85\% \\
\hline
Tests ejecutados & 1931 \\
\hline
\end{tabular}
\caption{Métricas de cobertura de pruebas con Jest}
\label{tab:jest-metrics}
\end{table}

\subsection{Pruebas Manuales en Web y PWA}
Además de estas pruebas automatizadas, se realizaron pruebas manuales tanto en la versión web como en la PWA\cite{pwa-webdev}. En esta etapa se revisó la navegación completa, el funcionamiento en dispositivos móviles, la instalación de la aplicación y los flujos principales de interacción desde el punto de vista del usuario final.
\subsection{Pruebas \textit{End-to-End} (E2E)}
Por último, se realizaron pruebas \textit{end-to-end} (E2E) para evaluar el funcionamiento del sistema completo, desde la interfaz hasta la base de datos. Este tipo de testing se centró en reproducir escenarios reales de uso, verificando que los distintos componentes interactúan de forma correcta y continua. Las pruebas E2E permitieron validar flujos completos como el inicio de sesión, la carga y consulta de datos, la comunicación con la API y la correcta actualización de la información en la base de datos. Este enfoque fue especialmente útil para detectar errores que no aparecen en pruebas aisladas, como problemas de integración, diferencias en el formato de los datos o comportamientos inesperados al combinar varias funcionalidades en un mismo flujo.
\subsection{Conclusión}
Gracias a esta combinación de pruebas unitarias, integradas, automatizadas, manuales y E2E, se logró obtener un sistema estable, con un buen nivel de calidad y sin fallos críticos al finalizar el desarrollo.

\section{La Solución Desarrollada}
\subsection{Objetivos de la Solución}
% Resumen de los objetivos que cumple la solución implementada.

\subsubsection{Gestión de Lotes de Semillas}
% Descripción del módulo de gestión de lotes.

\subsubsection{Análisis de Calidad}
% Describir procesos y cálculos asociados.

\subsubsection{Pureza Física}
% Detalles sobre la comprobación de pureza física.

\subsubsection{Germinación}
% Módulo y procesos de germinación.

\subsubsection{DOSN (Determinación de Otras Semillas en Número)}
% Explicar este cálculo y su implementación.

\subsubsection{PMS (Peso de Mil Semillas)}
% Cómo se calcula y registra.

\subsubsection{Tetrazolio}
% Proceso y registro de resultados.

\subsubsection{Flujo de Trabajo del Laboratorio}
% Describir el flujo automatizado/manual.

\subsubsection{Reportes y Exportaciones}
% Tipos de reportes y formatos de exportación soportados.

\section{Conclusiones}
El desarrollo del Sistema de Gestión de Laboratorio de Semillas (SGLS) para INIA Uruguay representa un avance clave en la modernización de los procesos de control de calidad. El proyecto surge para sustituir flujos de trabajo manuales basados en planillas independientes, que generaban dispersión de datos, dificultades en la trazabilidad y una alta dependencia del registro humano. Estas limitaciones afectaban la eficiencia diaria del laboratorio y podían comprometer la confiabilidad de los resultados, por lo que avanzar hacia un sistema centralizado y digital se volvió una necesidad estratégica.
Con la implementación del SGLS, se logró integrar en un único entorno todas las etapas del análisis de semillas, desde la recepción de muestras hasta la emisión de informes finales. Este enfoque unificado permitió eliminar la duplicación de datos, mejorar la trazabilidad de cada lote y asegurar un seguimiento claro y preciso a lo largo de todo el proceso. La automatización de cálculos complejos y la validación de información redujo considerablemente los errores humanos, aportando mayor consistencia y confianza en los resultados generados. Esto no solo agiliza el trabajo interno, sino que también fortalece la calidad de los servicios ofrecidos por el laboratorio.
Finalmente, el sistema no solo cumple con los requerimientos actuales del proceso de certificación, sino que establece una base sólida para el crecimiento futuro. Su diseño organizado y adaptable permite incorporar nuevos módulos, integrar equipamiento automatizado y ampliar las capacidades analíticas y de reporte que el laboratorio pueda necesitar más adelante. En conjunto, el SGLS posiciona a INIA como una institución alineada con las demandas modernas del sector agropecuario, demostrando cómo la digitalización puede potenciar la productividad, garantizar la calidad y respaldar la toma de decisiones con datos confiables.
\section{Trabajos a Futuro}
\subsection{Módulo de auditorías y control de calidad}
% Describir alcance y beneficios.

\subsection{Gestión documental avanzada}
% Ideas para mejorar la gestión documental.

\subsection{Automatización de cálculos y validaciones normativas}
% Automatizaciones pendientes o deseadas.

\subsection{Optimización del flujo de trabajo}
% Posibles mejoras en rendimiento o UX.

\subsection{Integración con sistemas externos}
% APIs y conectores deseados.


\section{Referencias}
% Bibliografía gestionada con BibTeX (estilo numérico)
% Crea/actualiza `bibliografia.bib` en el mismo directorio principal del proyecto.
\bibliographystyle{plain}
% Incluir todas las entradas de `bibliografia.bib` aunque no estén citadas
\nocite{*}
\bibliography{bibliografia}


\end{document}
% Documentación del Proyecto Final - Estilo visual APA (numerado)
\documentclass[12pt]{article}
% Idioma y codificación
\usepackage[spanish]{babel}
\usepackage[utf8]{inputenc}
\usepackage[T1]{fontenc}

% Tipografía y espaciado (estilo similar a APA)
\usepackage{mathptmx} % Times-like font (compatible)
\usepackage{setspace}
\doublespacing % APA suele usar interlineado doble

% Márgenes APA: 1 pulgada (2.54 cm)
\usepackage[letterpaper,margin=1in]{geometry}

% Encabezados y formato de secciones
\usepackage{fancyhdr}
\usepackage{titlesec}
\usepackage{graphicx}
\usepackage{enumitem}
\usepackage{hyperref}

% Configuración de encabezado (running head)
\pagestyle{fancy}
\fancyhf{}
\fancyhead[L]{\textit{Título corto del trabajo}}
\fancyhead[R]{\thepage}
\renewcommand{\headrulewidth}{0pt}


% Secciones estilo APA: flush left, bold, no extra vertical space

% (aseguramos que las macros estén correctamente escritas)
	itleformat{\section}{\normalfont\bfseries\large}{\thesection.}{0.5em}{}
	itleformat{\subsection}{\normalfont\bfseries}{\thesubsection.}{0.5em}{}

\setcounter{secnumdepth}{4}
\setcounter{tocdepth}{3}


	itle{Documentación del Proyecto Final}
\author{Nombre del Autor}
\date{\today}

\begin{document}
\maketitle
\vspace{1cm}
\tableofcontents
\clearpage

% Secciones principales (numeradas)

\section*{Agradecimientos}
\addcontentsline{toc}{section}{Agradecimientos}

Queremos expresar nuestro más sincero agradecimiento a aquellas personas que nos han acompañado y apoyado a lo largo de nuestra trayectoria académica. En especial, a nuestro docente y tutor Nicolás Escobar, quien ha sido un pilar fundamental en nuestra formación profesional. Sus observaciones, correcciones y sugerencias nos permitieron mejorar la calidad del proyecto y mantener una dirección clara en el proceso de desarrollo.

Agradecemos también a nuestras familias, por su apoyo constante y por facilitar las condiciones necesarias para llevar adelante este trabajo, especialmente durante los períodos de mayor dedicación. Su comprensión y acompañamiento hicieron posible sostener el ritmo de trabajo requerido.

A todas esas personas, nuestro más sincero agradecimiento.


\section*{Resumen}
\phantomsection
\addcontentsline{toc}{section}{Resumen}
El sistema web integral desarrollado para el Instituto Nacional de Investigación Agropecuaria (INIA) surge como una solución tecnológica para modernizar y digitalizar los procesos de análisis de calidad de semillas. El proyecto, realizado como trabajo final de la carrera Tecnólogo en Informática, aborda la necesidad de unificar y estandarizar procedimientos técnicos que anteriormente se gestionaban mediante planillas Excel.

El sistema permite administrar lotes y diversos análisis, como Germinación, Tetrazolio, PMS (Peso de Mil Semillas), DOSN (Determinación de Otras Semillas en Número) y Pureza Física, incorporando validaciones basadas en estándares internacionales, trazabilidad e historial técnico por lote, normalización de catálogos, importación de datos legados y generación de reportes especializados con exportación consolidada a Excel.

La arquitectura se implementó siguiendo un modelo cliente-servidor en capas, con un backend en Java Spring Boot 3.5 expuesto como API REST y reforzado con seguridad por JWT (RFC 7519) y autenticación en dos factores (2FA). Se incorporaron notificaciones en tiempo real mediante WebSockets, paginación \textit{offset} y \textit{limit}, y pruebas automatizadas con JUnit/JaCoCo. El frontend fue desarrollado en Next.js 14 con TypeScript, ofreciendo una PWA (Progressive Web App) optimizada y con un buen rendimiento.

El sistema resultante mejora la trazabilidad de los datos, la consistencia metodológica de los análisis y la eficiencia operativa institucional, estableciendo una base sólida para futuras ampliaciones y para la interoperabilidad con otros sistemas del INIA.

\section{Introducción}
En la última década, la digitalización ha transformado profundamente la gestión de información en sectores productivos y científicos, impulsando nuevos estándares de eficiencia, trazabilidad y precisión en los procesos operativos. En este contexto, el ámbito agropecuario no ha sido la excepción: la demanda de sistemas capaces de centralizar datos, reducir tareas manuales y garantizar la consistencia metodológica se ha vuelto fundamental para sostener la calidad de los análisis y la toma de decisiones institucionales.

Previo a este proyecto, el proceso de análisis de calidad de semillas en el Instituto Nacional de Investigación Agropecuaria (INIA) presentaba desafíos significativos derivados del uso de registros dispersos, planillas Excel independientes y procedimientos manuales con alto riesgo de error\cite{inia-uy}. Los análisis técnicos —como Germinación, Tetrazolio, PMS, DOSN y Pureza Física— se realizaban sobre lotes de semillas cuyos datos se registraban de forma heterogénea, dificultando el seguimiento del historial completo de cada lote\cite{ista2023,fao-seed}. Esta fragmentación generaba esfuerzos adicionales de normalización y verificación, con impacto en los tiempos de respuesta, en la trazabilidad de los resultados y en la comparabilidad histórica de la información técnica. Además, la ausencia de notificaciones oportunas y la falta de catálogos unificados limitaban la coordinación operativa entre equipos.

Frente a esta necesidad, surgió el desarrollo del Sistema Web Integral INIA, una plataforma diseñada para modernizar, centralizar y estandarizar todo el ciclo de vida de la información vinculada a los análisis de calidad de semillas. El proyecto, realizado como trabajo final de la carrera Tecnólogo en Informática, propone una solución que gestiona de manera estructurada los lotes, sus datos técnicos asociados, los contactos involucrados, las validaciones específicas de cada tipo de análisis, el historial completo de resultados y la generación automatizada de reportes. Asimismo, se incorporó la importación de información proveniente de sistemas legados, preservando datos históricos relevantes y facilitando la transición tecnológica.

En conjunto, esta solución tecnológica permitió profesionalizar la gestión de los análisis y de los lotes de semillas, reducir errores operativos, fortalecer la trazabilidad institucional y mejorar sustancialmente la eficiencia en el registro, control y procesamiento de información técnica. El sistema constituye una base sólida para futuras ampliaciones, nuevos módulos de interoperabilidad y una evolución continua hacia un ecosistema digital integral para el INIA.

\section{Objetivos Planteados y Resultados Esperados}
El principal objetivo del proyecto fue analizar, diseñar y desarrollar un sistema integral para la gestión de análisis de semillas realizados por el Instituto Nacional de Investigación Agropecuaria, orientado a la optimización del proceso actual de registro, edición, consulta y trazabilidad de las muestras (lotes) ingresadas para su análisis. Este sistema busca centralizar y modernizar la operativa ya existente, brindando tecnologías que faciliten la agilizar tareas, reducir errores y mejorar la disponibilidad de información para todos los posibles usuarios.
Durante el proceso de construcción del sistema se evaluaron diferentes alternativas tecnológicas para su implementación, optando por Java 21 junto con Springboot 3.5 y Spring Data JPA para el backend REST. Mientras que en el frontend se consideró React con Tailwind CSS 4.1. Asimismo se buscó la integración de módulos complementarios, como notificaciones, reportes y la exportación e importación de archivos formato xlsx.

\subsection{Objetivos específicos}
\begin{itemize}
  \item Un sistema centralizado y estable para gestionar todo el ciclo de análisis.
  \item Mayor trazabilidad y transparencia en el seguimiento de lotes.
  \item Reducción de errores mediante automatización y validaciones.
  \item Disminución del tiempo dedicado a tareas administrativas.
  \item Operativa más moderna, simple e intuitiva.
\end{itemize}

\section{Estado del Arte}
En los últimos años, la transformación digital se ha vuelto una realidad en prácticamente todas las organizaciones, tanto públicas como privadas. Este proceso busca mejorar la forma en que se gestionan los datos y se llevan adelante las tareas diarias. En el sector agropecuario, especialmente, la digitalización ha demostrado ser clave para lograr una mayor trazabilidad, precisión en los registros y eficiencia operativa. Por este motivo, cada vez más instituciones están dejando atrás los documentos en papel, las planillas de Excel y los procedimientos dispersos, y están migrando hacia plataformas web que centralizan y organizan la información de forma más segura y confiable. Esta tendencia está ampliamente respaldada por estudios académicos y por la práctica común de centros de investigación y laboratorios en todo el mundo.

\cite{inia-uy,ista2023,fao-seed}

El proyecto desarrollado para el INIA se inscribe en esta línea de evolución tecnológica. La institución necesita una herramienta digital que permita gestionar sus análisis de forma integral: registrar y consultar muestras (lotes), administrar configuraciones, controlar distintos roles de usuario, generar reportes, exportar información y también incorporar datos históricos provenientes de sistemas anteriores. Este capítulo presenta el marco conceptual, tecnológico y comparativo que justifica la solución propuesta y contextualiza su diseño dentro de las tendencias actuales de la industria del software.

\cite{inia-uy}
\subsection{Marco Teórico}
En estos últimos años las organizaciones adoptaron hojas de cálculo como herramienta base para almacenar información gracias a su accesibilidad, bajo costo, flexibilidad y facilidad de uso. Sin embargo, a medida que los procesos se vuelven más complejos y los volúmenes de datos crecen, estos mecanismos se vuelven insuficientes y arriesgados para labores críticas. Las limitaciones más comunes incluyen:
\begin{itemize}
    \item \textbf{Escalabilidad restringida:} Las hojas de cálculo no están diseñadas para manejar grandes volúmenes de datos ni crecer de forma sostenible. A medida que aumentan los registros, las operaciones se vuelven lentas, propensas a fallos y difíciles de mantener.
    \item \textbf{Colaboración limitada:} La funcionalidad de edición simultánea que es ofrecida por la mayoría de las aplicaciones de gestión de hojas de cálculo es vulnerable a conflictos, sobreescrituras y pérdida de información.
    \item \textbf{Falta de trazabilidad y mecanismos de auditoría:} Resulta difícil rastrear cambios, identificar responsables y asegurar integridad de datos.
    \item \textbf{Integración deficiente con sistemas externos:} La conexión con APIs, bases de datos, servicios externos u otros sistemas institucionales resulta limitada.
\end{itemize}
\subsection{La Transición Hacia Sistemas Web}
Las planillas suelen funcionar bien en etapas iniciales, pero a medida que crecen los volúmenes de información o los procesos se vuelven más complejos, es común que se busquen alternativas web que permitan centralizar la información y manejarla de manera más eficiente. Los sistemas modernos ofrecen:
\begin{itemize}
    \item Acceso multiplataforma.
    \item Escalabilidad horizontal mediante arquitecturas distribuidas.
    \item Integración nativa a través de APIs REST.
    \item Validaciones automáticas a nivel de negocio.
    \item Auditorías completas de operaciones.
    \item Estandarización de procesos y flujos de trabajo.
\end{itemize}
\cite{ista2023,fao-seed}
Este enfoque tecnológico es el más utilizado en laboratorios, centros de investigación y organizaciones científicas para gestionar muestras, controles de calidad, análisis y trazabilidad de procesos experimentales.
\subsection{Tecnologías}
El proyecto adopta tecnologías modernas, robustas y ampliamente utilizadas en la industria. La selección busca equilibrio entre estabilidad, madurez, facilidad de mantenimiento y alineación con estándares profesionales.
\cite{spring-boot,nextjs}
\subsubsection{Frontend: React + TypeScript}
React es uno de los frameworks más utilizados globalmente para construir interfaces de usuario debido a:
\begin{itemize}
    \item Su modelo de componentes reutilizables
    \item Su alto rendimiento mediante virtual DOM
    \item Un ecosistema amplio de librerías
    \item Facilidad para construir SPA (Single Page Applications)
\end{itemize}
El uso de TypeScript aporta tipado estático, reduce errores y mejora la mantenibilidad del código.
\subsubsection{Backend: Java + Spring Boot}
Spring Boot es estándar en el desarrollo empresarial gracias a:
\begin{itemize}
    \item Integración nativa con Spring Security
    \item Soporte simplificado para APIs REST
    \item Inyección de dependencias y modularidad
    \item Comunidad madura y documentación extensa
\end{itemize}
Java 21, como versión LTS, asegura estabilidad a largo plazo.
\subsubsection{Base de Datos: PostgreSQL}
Elegida por ser:
\begin{itemize}
    \item Open source
    \item Altamente confiable
    \item Compatible con operaciones complejas
    \item Escalable y sólida para manejo de datos institucionales
\end{itemize}
\subsubsection{Infraestructura y Arquitectura}
\begin{itemize}
    \item Modelo cliente-servidor (frontend Next.js – backend Spring Boot)
    \item Arquitectura en tres capas (presentación – servicios – datos)
    \item Uso de Docker para portabilidad y despliegue
    \item Estructura modular por dominios (análisis, seguridad, notificaciones, usuarios)
    \item Comunicación vía REST y WebSocket (notificaciones en tiempo real) \cite{rfc6455,rfc7519}
\end{itemize}
\subsection{Soluciones Similares}
En el mercado existen herramientas orientadas a gestionar información estructurada, colaborar en equipos y reemplazar flujos basados en planillas. Aunque ninguna se adapta exactamente a los procesos complejos del INIA, sirven como referencia sobre cómo la industria resuelve problemas similares.
\subsubsection{Airtable}
Airtable combina conceptos de base de datos con la interfaz amigable de una hoja de cálculo. Es una plataforma low-code orientada a pequeños proyectos y equipos que necesitan digitalizar procesos sin desarrollar software propio.

Ventajas:
\begin{itemize}
    \item Interfaz simple y accesible
    \item Colaboración en tiempo real
    \item APIs integradas
    \item Automatizaciones básicas
\end{itemize}
Limitaciones:
\begin{itemize}
    \item No escala para flujos complejos
    \item Restricciones para reglas de negocio avanzadas
    \item Dependencia de licencias externas
\end{itemize}
\cite{airtable}
\begin{figure}[htbp]
    \centering
    \includegraphics[width=0.8\textwidth,height=0.6\textheight,keepaspectratio]{Airtable.png}
    \caption{Ejemplo: vista de contenido en Airtable (tabla/registro).}
    \label{fig:airtable}
\end{figure}
\subsubsection{Smartsheet}
Smartsheet ofrece una experiencia similar a Excel, pero con mayor control, trazabilidad y herramientas para flujos de trabajo.

Ventajas:
\begin{itemize}
    \item Gestión de proyectos y procesos
    \item Reportes avanzados
    \item Automatizaciones integradas
\end{itemize}
Limitaciones:
\begin{itemize}
    \item Alto costo según uso
    \item Menor flexibilidad frente a un desarrollo a medida
    \item Dependencia del ecosistema propietario
\end{itemize}
\cite{smartsheet}
\begin{figure}[htbp]
    \centering
    \includegraphics[width=0.8\textwidth,height=0.6\textheight,keepaspectratio]{Smartsheet.png}
    \caption{Ejemplo: interfaz de Smartsheet mostrando una vista tipo hoja de cálculo y reportes.}
    \label{fig:smartsheet}
\end{figure}
\subsubsection{Odoo}
Odoo es un ERP (Planificador de Recursos Empresariales) modular que integra distintos dominios empresariales.

Ventajas:
\begin{itemize}
    \item Gran variedad de módulos
    \item Comunidad activa
    \item Escalabilidad para múltiples áreas
\end{itemize}
Limitaciones:
\begin{itemize}
    \item Instalación y configuración complejas
    \item Sobrecarga funcional para proyectos específicos
    \item Dificultad de adaptación a procesos científicos y de laboratorio
\end{itemize}
\cite{odoo}
\begin{figure}[htbp]
    \centering
    \includegraphics[width=0.8\textwidth,height=0.6\textheight,keepaspectratio]{Odoo.png}
    \caption{Ejemplo: módulo de Odoo con vistas integradas de gestión de recursos.}
    \label{fig:odoo}
\end{figure}
\subsection{Conclusión del análisis comparativo}
Si bien las herramientas ofrecen funcionalidades útiles, no cumplen de forma precisa con los requerimientos del INIA, especialmente en lo relativo a:
\begin{itemize}
    \item Procesos de laboratorio
    \item Gestión de análisis y múltiples roles
    \item Validaciones específicas
    \item Importación de datos legados
    \item Adaptación a flujos técnicos
\end{itemize}
Por ello, un sistema a medida es la opción más adecuada para atender las necesidades reales y garantizar la adaptación total a los procesos internos y ofrecer la flexibilidad necesaria para futuros cambios o expansiones.
\cite{inia-uy,ista2023}
\section{Análisis del Problema}
Para poder desarrollar el sistema web del INIA fue necesario entender a fondo cuáles eran los problemas reales en el trabajo diario del laboratorio. Durante mucho tiempo los análisis se registraban en planillas haciendo difícil mantener un control claro y asegurar la uniformidad de la información. Este diagnóstico inicial fue clave para diseñar una solución que realmente mejorase la forma en que los análisis se gestionan y resultase útil para el cliente.
A partir de este análisis, se estudiaron cuidadosamente los procesos actuales, se conversó con los usuarios y se documentaron sus necesidades. Esto permitió definir cómo debía funcionar el sistema, qué información era importante, quiénes interactúan con él y qué herramientas debían construirse para hacer su trabajo más sencillo y seguro. Con esta información se estableció el alcance del proyecto y se identificaron los puntos críticos a resolver en cuanto a diseño y desarrollo.
\subsection{Vista del Modelo de Dominio}
El modelo de dominio reúne todas las entidades del sistema y muestra cómo se relacionan todas entre sí. Tener esta vista clara fue fundamental para ordenar las ideas, entender el flujo de la información y asegurarse de que el sistema reflejase fielmente la forma en que se trabaja en el laboratorio.
\begin{figure}[htbp]
    \centering
    \includegraphics[width=0.8\textwidth,height=0.6\textheight,keepaspectratio]{\detokenize{Diagrama de Dominio.png}}
    \caption{Modelo de dominio: entidades y relaciones principales.}
    \label{fig:modelo_dominio}
\end{figure}
\subsection{Definición de Casos de Uso}
Los casos de uso ayudan a describir qué puede hacer cada usuario y cómo se relacionan las acciones con las funcionalidades disponibles en el sistema. A partir de ellos se identificaron tareas clave como crear y gestionar lotes y análisis, administrar catálogos, generar reportes y cargar datos históricos.
\subsection{Actores}
Los actores son los elementos externos al sistema, ya sean usuarios o sistemas, que interactúan con la plataforma para llevar a cabo determinadas tareas. En este proyecto se definieron tres actores centrales:

\textbf{Administrador:}
\begin{itemize}
    \item Administra catálogos, lista de contactos y aprobación de usuarios.
    \item Gestiona análisis y lotes.
    \item Importa datos históricos.
    \item Otorga y revoca permisos.
    \item Exporta reportes en formato xlsx.
    \item Supervisa y valida las acciones realizadas por los Analistas.
    \item Observa reportes.
    \item Tiene acceso total a todos los módulos del sistema.
\end{itemize}
\textbf{Analista:}
\begin{itemize}
    \item Registra y edita lotes y análisis.
    \item Carga y modifica información técnica.
    \item Requiere la aprobación del Administrador para confirmar los resultados de un análisis.
    \item Exporta reportes en formato xlsx.
    \item Observa reportes.
\end{itemize}
\textbf{Observador:}
\begin{itemize}
    \item Solo puede visualizar información.
    \item Accede a lotes, análisis, resultados y reportes.
    \item No posee permisos para modificar datos.
\end{itemize}
\subsection{Diagrama de Casos de Uso}
El siguiente diagrama muestra los principales casos de uso del sistema:
\begin{figure}[htbp]
    \centering
    \includegraphics[width=0.8\textwidth,height=0.6\textheight,keepaspectratio]{\detokenize{Diagrama Casos de Uso.png}}
    \caption{Diagrama de casos de uso: principales interacciones entre actores y el sistema.}
    \label{fig:casos_uso}
\end{figure}
\subsection{Vista del Modelo de Diseño}
El diseño propuesto para la arquitectura del sistema busca mantener un equilibrio claro entre simplicidad, organización y capacidad de crecimiento. La estructura en capas responde a la necesidad de separar responsabilidades y asegurar que cada parte del sistema pueda evolucionar sin generar impacto innecesario en las demás. Este enfoque no solo mejora la mantenibilidad, sino que también facilita la incorporación de nuevas funcionalidades en el futuro. 
La organización del sistema se basa en una arquitectura por capas que incluye la capa de Cliente, donde se ubica la aplicación web, incluyendo sus capacidades de PWA, y la capa de Presentación, encargada de gestionar la interfaz y la comunicación inicial con el backend. A esto se suma la capa de Seguridad, que centraliza los mecanismos de autenticación, autorización y protección de datos. Por su parte, la capa de Aplicación, implementada con Spring Boot, contiene toda la lógica del negocio y se encarga del flujo de información hacia la capa de Datos, responsable del acceso y persistencia.
En esta arquitectura, los servicios adicionales que utiliza el sistema, como el envío de correos o la verificación en dos pasos, se integran directamente dentro de la capa de Aplicación. Esto evita complejidades innecesarias en la representación del diseño y mantiene el diagrama coherente y simple, sin dejar de reflejar el funcionamiento real del sistema.
\begin{figure}[htbp]
    \centering
    \includegraphics[width=0.8\textwidth,height=0.6\textheight,keepaspectratio]{\detokenize{Diagrama de Arquitectura.png}}
    \caption{Diagrama de arquitectura: capas y componentes principales del sistema.}
    \label{fig:arquitectura}
\end{figure}
\subsection{Descripción de la Arquitectura del Sistema}
La arquitectura propuesta organiza el sistema en capas bien definidas, lo que permite mantener una estructura ordenada, escalable y fácil de mantener. Cada capa cumple un rol específico dentro del flujo general de la aplicación.
\subsubsection{Capa de Cliente (SPA/PWA)}
El frontend funciona como una aplicación web moderna basada en Next.js y diseñada como SPA con capacidades PWA. Desde el navegador, los usuarios interactúan mediante una interfaz rápida, responsiva y adaptable a distintos dispositivos. La aplicación soporta instalación como app y toda la comunicación con el backend se realiza a través de REST.
\subsubsection{Capa de Seguridad}
Incluye los mecanismos que protegen el acceso al sistema. Se utiliza autenticación basada en JWT almacenado en cookies seguras y un sistema de doble factor (2FA) mediante códigos TOTP compatibles con Google Authenticator. La autorización se gestiona con Spring Security mediante roles como ADMIN, ANALISTA y OBSERVADOR. El frontend también aporta seguridad con middleware que controla el acceso a rutas protegidas. Además, existe un sistema de notificaciones por correo para avisos de seguridad y recuperación de cuentas.
\subsubsection{Capa de Presentación (Frontend)}
Esta capa engloba la lógica de presentación, los componentes visuales, la validación de formularios y el manejo de estado. Se utilizan herramientas como React Query, React Hook Form, Zod, Radix UI, shadcn/ui y Tailwind. El frontend organiza sus rutas y funcionalidades en módulos bien definidos: autenticación, administración, análisis, reportes, perfil, notificaciones, etc. También ofrece integración con WebSockets para recibir actualizaciones y notificaciones automáticamente e incluye funcionalidades como dashboards interactivos, listados, formularios complejos e instalación como PWA.
\subsubsection{Capa de Aplicación (Backend)}
El backend está construido con Spring Boot siguiendo el patrón MVC y estructurado en Controllers, Services y Repositories. Los Controllers exponen las APIs, los Services contienen la lógica de negocio y los Repositories gestionan el acceso a la base de datos. Aquí se manejan análisis de semillas, validaciones, reportes, importación de datos históricos, seguridad, notificaciones y todo el flujo del sistema. También se implementa un canal WebSocket con STOMP para notificaciones en tiempo real, manejo de transacciones y un sistema global de manejo de excepciones.
\subsubsection{Capa de Datos}
La persistencia está implementada con Spring Data JPA e Hibernate, usando PostgreSQL como base de datos. La base está diseñada con relaciones bien definidas e integridad referencial.
\subsection{Descomposición en subsistemas}
El sistema se organiza en cinco subsistemas principales que trabajan en conjunto para ofrecer una plataforma robusta, segura y orientada a la experiencia del usuario. Cada uno cumple un rol específico dentro de la solución, y en conjunto conforman una arquitectura coherente y fácil de mantener. A continuación, se detalla cada subsistema y los elementos que lo componen.
\subsubsection{Subsistema de Cliente}
Representa lo que usa el usuario en su navegador o como PWA instalada.
\begin{itemize}
    \item Renderiza la interfaz y maneja la experiencia visual e interacción.
    \item Funciona incluso con conexión inestable
    \item Administra cookies de sesión y comunicación con el backend vía HTTPS y WebSockets.
    \item Se encarga de notificaciones push y la instalación como app en dispositivos.
\end{itemize}
\subsubsection{Subsistema de Seguridad}
\begin{itemize}
    \item Gestiona la autenticación con JWT y Cookies HttpOnly.
    \item Implementa 2FA con TOTP y manejo de dispositivos confiables.
    \item Controla los permisos según roles: ADMIN, ANALISTA, OBSERVADOR.
    \item Administra contraseñas, recuperación de acceso y validaciones.
    \item Incluye middleware que protege rutas y verifica sesiones en el frontend.
    \item Genera alertas por eventos críticos de seguridad (cambios de clave).
\end{itemize}
\subsubsection{Subsistema de Presentación (Frontend)}
\begin{itemize}
    \item SPA/PWA construida con componentes reutilizables.
    \item Renderiza pantallas, formularios, tablas, flujos y navegación interna.
    \item Maneja el estado global de la aplicación.
    \item Canaliza toda la comunicación hacia el backend mediante servicios API.
    \item Recibe actualizaciones en tiempo real mediante WebSockets.
    \item Aplica reglas de visibilidad según rol del usuario.
\end{itemize}
\subsubsection{Subsistema de Aplicación (Backend)}
\begin{itemize}
    \item Contiene la lógica de negocio principal del sistema.
    \item Expone endpoints REST desde los controllers.
    \item Ejecuta reglas de negocio, validaciones e integraciones internas.
    \item Administra transacciones y operaciones complejas de múltiples pasos.
    \item Gestiona módulos como catálogos, reportes, usuarios, análisis, lotes, etc.
    \item Envía notificaciones en tiempo real al frontend.
    \item Devuelve respuestas seguras, consistentes y auditables.
\end{itemize}
\subsubsection{Subsistema de Datos}
\begin{itemize}
    \item Gestiona el acceso a la base de datos PostgreSQL.
    \item Define las entidades del dominio y su estructura.
    \item Implementa repositorios JPA para consultas y persistencia.
    \item Garantiza integridad, consistencia y trazabilidad de la información.
    \item Registra auditoría automática de creación y modificación de registros.
\end{itemize}
\section{Implementación}
\subsection{Diagrama de \textit{Deployment} de UML}
El diagrama de \textit{Deployment} permite mostrar de manera clara cómo quedó distribuida la arquitectura física del sistema una vez implementado. Allí se identificaron los distintos nodos de hardware y software que intervinieron en la solución, así como la forma en que se organizaron los componentes dentro de la infraestructura disponible. El sistema se desplegó en un servidor virtual de Amazon Web Services (AWS), utilizando una instancia EC2 configurada para ejecutar contenedores Docker. En este entorno se alojaron los servicios del frontend, el backend y la base de datos, los cuales se comunicaron a través de una red interna propia del host.

 El diagrama también reflejó la forma en que los usuarios accedieron a la aplicación, ya fuera desde un navegador web o desde la versión móvil como PWA\cite{pwa-webdev}. En ambos casos, la conexión se estableció a través de Internet hacia la dirección pública del servidor EC2. Gracias a esta representación fue posible visualizar de manera sintética cómo se relacionaron los distintos elementos de la solución y cómo se organizó el entorno de despliegue, lo que sirvió como punto de partida para el análisis detallado de las tecnologías utilizadas en el backend y el frontend.
\cite{inia-uy,aws}
\begin{figure}[htbp]
	\centering
	% Mostrar el diagrama de deployment ocupando la mayor área de la página
	\includegraphics[width=\textwidth,height=0.92\textheight,keepaspectratio]{\detokenize{Deploy UML.png}}
	\caption{Diagrama de \textit{deployment} (UML) que muestra la distribución de nodos y contenedores en la infraestructura de despliegue.}
	\label{fig:deploy}
\end{figure}
\subsection{Tecnologías utilizadas}
\subsubsection{Cliente}
El cliente del sistema fue diseñado como una aplicación web dinámica utilizando Next.js 14 y React 18, tecnologías ampliamente adoptadas en entornos productivos debido a su eficiencia, modularidad y soporte comunitario.
\cite{nextjs}

La estructura se organizó siguiendo las convenciones del App Router de Next.js, lo cual permitió una clara separación entre páginas, componentes y módulos de lógica, promoviendo escalabilidad y mantenibilidad.

Para el desarrollo de la interfaz de usuario se utilizaron Tailwind CSS, Radix UI y Shadcn. Estas tecnologías permitieron crear una experiencia visual coherente y accesible, con componentes reutilizables y mantenibles, alineados con estándares modernos de diseño.
\cite{tailwindcss,radixui,shadcn}

La comunicación entre el frontend y el backend se realizó mediante APIs REST para la mayor parte de las operaciones, mientras que las funcionalidades en tiempo real se implementaron a través de WebSocket, permitiendo la recepción de notificaciones instantáneas y mejorando la experiencia interactiva del usuario.

El aseguramiento de calidad del frontend incluyó la ejecución de pruebas automatizadas mediante Jest y React Testing Library, herramientas ampliamente utilizadas para validar el comportamiento de componentes y flujos de interacción.
\cite{jest,rtl}
\subsubsection{Servidor}
El backend se implementó utilizando Java 21 con Spring Boot 3.5, aplicando una arquitectura en capas que divide la lógica de negocio, los servicios, los controladores web, los repositorios de datos y las configuraciones transversales. Esta estructura facilita el mantenimiento, reduce el acoplamiento y mejora la estabilidad del sistema.
\cite{spring-boot}
\subsubsection{Servicios y API}
La API del sistema fue documentada utilizando OpenAPI/Swagger, permitiendo que tanto desarrolladores como actores externos comprendan las especificaciones de cada endpoint, formatos de datos, tipos de respuestas y códigos de error.
\cite{openapi}

El backend expone servicios REST y un canal de comunicación en tiempo real mediante WebSocket (STOMP), empleado para enviar notificaciones y eventos relevantes sin necesidad de realizar consultas repetitivas al servidor.
\subsubsection{Persistencia y Base de Datos}
La aplicación utiliza PostgreSQL 15, gestionado mediante Spring Data JPA, lo cual permitió automatizar tareas de persistencia y reducir la necesidad de escribir consultas SQL manuales.
\cite{postgresql}

Durante el proceso de pruebas se utilizó H2, una base de datos en memoria que permite ejecutar los tests de forma rápida, aislada y sin requerir infraestructura externa.
\subsubsection{Seguridad}
La seguridad del sistema se abordó mediante:
\begin{itemize}
    \item Spring Security para la protección de endpoints y control de roles.
    \item JWT para la gestión de sesiones sin estado.
    \item Autenticación en dos pasos (2FA) mediante códigos TOTP, fortaleciendo la seguridad del acceso.
\end{itemize}
\cite{spring-security,rfc7519,rfc6238}

    Este conjunto de herramientas permitió cumplir buenas prácticas de seguridad alineadas con estándares modernos.
\subsubsection{Herramientas de Construcción, Testing y DevOps}
El backend fue compilado y gestionado mediante Maven, lo que permitió automatizar tareas de instalación, test y empaquetado.

La calidad del código se reforzó mediante pruebas unitarias utilizando JUnit 5 y Mockito, enfocadas en validar la lógica de negocio, la interacción entre módulos y el correcto comportamiento de los componentes críticos.
\cite{junit5,jacoco}

Finalmente, el uso de Docker y Docker Compose permitió ejecutar el proyecto de forma completa (frontend, backend y base de datos) mediante entornos aislados y reproducibles, asegurando coherencia en las diferentes etapas del desarrollo.
\cite{docker}

La comunicación con la base de datos se realizó a través de JPA/Hibernate, lo que permitió manejar las entidades del dominio de forma eficiente y reducir la carga de escribir consultas SQL manuales.\cite{hibernate} El servicio funcionó dentro de un contenedor Docker, lo que facilitó su despliegue en la instancia EC2 y garantiza un entorno consistente durante todo el ciclo de vida del proyecto.
\subsection{Horas Dedicadas}
A continuación se detallarán los registros de horas por etapa del proyecto. Estos fueron gestionados utilizando un proyecto en Toggl.
\cite{toggl}

% Resumen rápido de horas por etapa
\begin{center}
\begin{tabular}{l r}
\hline
	\textbf{Etapa} & \textbf{Horas} \\
\hline
Etapa 1 & 60 \\
Etapa 2 & 553 \\
Etapa 3 & 86 \\
\hline
\end{tabular}
\end{center}

% Nota: Para que \texttt{\textcolor{red}} funcione, asegúrate de cargar el paquete \texttt{xcolor} en el preámbulo si aún no está incluido.
\subsubsection{Etapa 1: Documentación y Análisis}
\textbf{Período:} Desde el 14 de agosto al 08 de septiembre de 2025.
La primera etapa se enfocó principalmente en recolectar la información pertinente con las clientes y documentar las características inicialmente establecidas para comenzar con el desarrollo adecuado de la aplicación web.

Un total de 60 horas fueron dedicadas a esta fase.
\subsubsection{Etapa 2: Desarrollo}
\textbf{Período:} Desde el 08 de septiembre al 08 de noviembre de 2025.
La etapa de desarrollo se dedicó enteramente al desarrollo completo de la aplicación web, cubriendo los requerimientos solicitados y validando con las clientes.

Un total de 553 horas fueron dedicadas a esta fase.
\subsubsection{Etapa 3: Testing, Documentación Final y Presentación}
\textbf{Período:} Desde el 08 de noviembre al 30 de noviembre de 2025.
La etapa final del proyecto se enfocó en la realización de pruebas funcionales y de usabilidad, así como en la corrección de errores detectados. Además, se completó la documentación técnica y se preparó la presentación final del proyecto.

Un total de 86 horas fueron dedicadas a esta fase.
\section{Gestión de Proyecto}
Para la ejecución de este proyecto no se estructuraron roles específicos para cada integrante del equipo. Si bien cada miembro tenía mayor afinidad con determinadas áreas tecnológicas, se decidió adoptar una dinámica de trabajo colaborativa en la que todos participaron tanto en el desarrollo del frontend como el backend, así como en tareas vinculadas a documentación, testing y despliegue.
\cite{inia-uy}

Para la organización del trabajó se creó un proyecto en Notion, donde se registraron las tareas principales, los responsables y las estimaciones de tiempo para cada tarea. 
\cite{notion}

El principal medio de comunicación en línea fue Google Meet, utilizado para reuniones virtuales, donde a su vez se llevaron a cabo reuniones periódicas con el cliente. Además, se creó un grupo de WhatsApp para coordinar rápidamente temas urgentes, notificaciones internas y actualizaciones cortas entre los integrantes del equipo, lo que permitió una comunicación ágil y continua.
\cite{googlemeet,whatsapp}
\subsection{Entorno de Desarrollo}
El desarrollo del sistema se realizó en un entorno orientado a aplicaciones web modernas, priorizando la reproducibilidad, la trazabilidad y la correcta separación de responsabilidades entre los distintos componentes.
\cite{docker}

Para el desarrollo del software se utilizaron dos entornos de desarrollo integrados (IDE):
\begin{itemize}
    \item Visual Studio Code, empleado principalmente para el frontend y configurado con extensiones específicas para TypeScript, React, Docker y herramientas de control de versiones.
    \cite{vscode}
    \item IntelliJ IDEA, utilizado para el backend en Java, aprovechando su soporte avanzado para Spring Boot, gestión de dependencias con Maven y depuración integrada.
    \cite{intellij}
\end{itemize}
La gestión del código fuente se realizó utilizando Git como sistema de control de versiones y GitHub como plataforma principal de alojamiento, revisión y seguimiento del proyecto. GitHub permitió manejar ramas, realizar revisiones de código, gestionar issues y mantener un flujo de trabajo organizado basado en buenas prácticas de versionado.
\cite{git,github}

Esta combinación permitió a todos los integrantes del equipo trabajar de forma eficiente y con herramientas adecuadas para cada tecnología.

\vspace{1em}
\noindent\textbf{Nota:} Para información detallada sobre la identificación, evaluación y mitigación de riesgos del proyecto, consultar el \textbf{Anexo 3 -- Mitigación de Riesgos} en el documento \texttt{Anexos.pdf}.

\section{Problemas Encontrados}
\subsection{Reestructuración del Módulo de Germinación}
Durante la etapa final del desarrollo se identificó un problema crítico en el módulo de germinación, que obligó a replantear su estructura, lógica interna y funcionamiento general. Este inconveniente surgió a raíz de una discrepancia significativa entre los requisitos originalmente planteados y las necesidades reales del laboratorio\cite{inia-uy,ista2023}.

La dificultad se originó en la documentación inicial proporcionada por el cliente, compuesta por hojas de cálculo con información incompleta, las cuales eran utilizadas por el equipo para realizar la gestión de forma manual. A partir de la referencia el equipo asumió que todas las fechas de conteo eran comunes a todo el análisis de germinación y que las diferentes configuraciones de días de prefrío ingresadas en el mismo análisis no influían en las mismas, entre otras\cite{fao-seed,ista2023}.

Estas suposiciones llevaron a diseñar el módulo bajo una estructura que no reflejaba la complejidad real del proceso.
\subsubsection{Cambio de Requisitos}
Pocos días antes de la finalización de la etapa de desarrollo, el cliente aclaró que el funcionamiento real del análisis de germinación difería de lo inicialmente interpretado. Los requisitos correctos incluían configuraciones múltiples con fechas de conteos independientes, restringidas por los días de prefrío y con validaciones específicas adicionales.

Este cambio presentó un ajuste conceptual profundo respecto a la estructura inicial del módulo, y su corrección requirió una reingeniería completa del módulo, afectando múltiples áreas del sistema. Entre las acciones necesarias se incluyó el rediseño de la estructura de los datos, reescritura de la lógica del sistema y modificación de interfaces.

A pesar de las restricciones de tiempo, el equipo logró implementar correctamente la nueva estructura, la cual resultó ser fundamental para asegurar que el módulo cumpliera adecuadamente con los cambios solicitados por el cliente\cite{inia-uy}.
\input{sections/10-solucion_desarrollada.tex}
\input{sections/11-conclusiones.tex}
\input{sections/12-trabajos_futuro.tex}
\input{sections/13-referencias.tex}

\end{document}
