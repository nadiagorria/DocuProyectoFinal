\section*{Resumen}
\addcontentsline{toc}{section}{Resumen}
El sistema web integral desarrollado para el Instituto Nacional de Investigación Agropecuaria (INIA) surge como una solución tecnológica para modernizar y digitalizar los procesos de análisis de calidad de semillas\cite{inia-uy,ista2023}. El proyecto, realizado como trabajo final de la carrera Tecnólogo en Informática, aborda la necesidad de unificar y estandarizar procedimientos técnicos que anteriormente se gestionaban mediante planillas Excel.

El sistema permite administrar lotes y diversos análisis, como Germinación, Tetrazolio, PMS (Peso de Mil Semillas), DOSN (Determinación de Otras Semillas en Número) y Pureza Física, incorporando validaciones basadas en estándares internacionales\cite{ista2023,fao-seed}, trazabilidad e historial técnico por lote, normalización de catálogos, importación de datos legados y generación de reportes especializados con exportación consolidada a Excel.

La arquitectura se implementó mediante microservicios, con un backend en Java Spring Boot 3.5 expuesto como API REST\cite{spring-boot} y reforzado con seguridad por JWT (RFC 7519)\cite{rfc7519} y autenticación en dos factores (2FA). Se incorporaron notificaciones en tiempo real mediante WebSockets\cite{rfc6455}, paginación basada en cursores y pruebas automatizadas con JUnit/JaCoCo\cite{junit5,jacoco}. El frontend fue desarrollado en Next.js 14 con TypeScript\cite{nextjs}, ofreciendo una PWA (Progressive Web App) optimizada y con un buen rendimiento.

El sistema resultante mejora la trazabilidad de los datos, la consistencia metodológica de los análisis y la eficiencia operativa institucional, estableciendo una base sólida para futuras ampliaciones y para la interoperabilidad con otros sistemas del INIA.
