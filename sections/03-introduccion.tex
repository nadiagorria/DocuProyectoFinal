\section{Introducción}
En la última década, la digitalización ha transformado profundamente la gestión de información en sectores productivos y científicos, impulsando nuevos estándares de eficiencia, trazabilidad y precisión en los procesos operativos. En este contexto, el ámbito agropecuario no ha sido la excepción: la demanda de sistemas capaces de centralizar datos, reducir tareas manuales y garantizar la consistencia metodológica se ha vuelto fundamental para sostener la calidad de los análisis y la toma de decisiones institucionales.

Previo a este proyecto, el proceso de análisis de calidad de semillas en el Instituto Nacional de Investigación Agropecuaria (INIA) presentaba desafíos significativos derivados del uso de registros dispersos, planillas Excel independientes y procedimientos manuales con alto riesgo de error\cite{inia-uy}. Los análisis técnicos —como Germinación, Tetrazolio, PMS, DOSN y Pureza Física— se realizaban sobre lotes de semillas cuyos datos se registraban de forma heterogénea, dificultando el seguimiento del historial completo de cada lote\cite{ista2023,fao-seed}. Esta fragmentación generaba esfuerzos adicionales de normalización y verificación, con impacto en los tiempos de respuesta, en la trazabilidad de los resultados y en la comparabilidad histórica de la información técnica. Además, la ausencia de notificaciones oportunas y la falta de catálogos unificados limitaban la coordinación operativa entre equipos.

Frente a esta necesidad, surgió el desarrollo del Sistema Web Integral INIA, una plataforma diseñada para modernizar, centralizar y estandarizar todo el ciclo de vida de la información vinculada a los análisis de calidad de semillas. El proyecto, realizado como trabajo final de la carrera Tecnólogo en Informática, propone una solución que gestiona de manera estructurada los lotes, sus datos técnicos asociados, los contactos involucrados, las validaciones específicas de cada tipo de análisis, el historial completo de resultados y la generación automatizada de reportes. Asimismo, se incorporó la importación de información proveniente de sistemas legados, preservando datos históricos relevantes y facilitando la transición tecnológica.

En conjunto, esta solución tecnológica permitió profesionalizar la gestión de los análisis y de los lotes de semillas, reducir errores operativos, fortalecer la trazabilidad institucional y mejorar sustancialmente la eficiencia en el registro, control y procesamiento de información técnica. El sistema constituye una base sólida para futuras ampliaciones, nuevos módulos de interoperabilidad y una evolución continua hacia un ecosistema digital integral para el INIA.
