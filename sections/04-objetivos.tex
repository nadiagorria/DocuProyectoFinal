\section{Objetivos Planteados y Resultados Esperados}
El principal objetivo del proyecto fue analizar, diseñar y desarrollar un sistema integral para la gestión de análisis de semillas realizados por el Instituto Nacional de Investigación Agropecuaria, orientado a la optimización del proceso actual de registro, edición, consulta y trazabilidad de las muestras (lotes) ingresadas para su análisis. Este sistema busca centralizar y modernizar la operativa ya existente, brindando tecnologías que faciliten la agilizar tareas, reducir errores y mejorar la disponibilidad de información para todos los posibles usuarios.
Durante el proceso de construcción del sistema se evaluaron diferentes alternativas tecnológicas para su implementación, optando por Java 21 junto con Springboot 3.5 y Spring Data JPA para el backend REST. Mientras que en el frontend se consideró React con Tailwind CSS 4.1. Asimismo se buscó la integración de módulos complementarios, como notificaciones, reportes y la exportación e importación de archivos formato xlsx.

\subsection{Objetivos específicos}
\begin{itemize}
  \item Un sistema centralizado y estable para gestionar todo el ciclo de análisis.
  \item Mayor trazabilidad y transparencia en el seguimiento de lotes.
  \item Reducción de errores mediante automatización y validaciones.
  \item Disminución del tiempo dedicado a tareas administrativas.
  \item Operativa más moderna, simple e intuitiva.
\end{itemize}
