\section{Estado del Arte}
En los últimos años, la transformación digital se ha vuelto una realidad en prácticamente todas las organizaciones, tanto públicas como privadas. Este proceso busca mejorar la forma en que se gestionan los datos y se llevan adelante las tareas diarias. En el sector agropecuario, especialmente, la digitalización ha demostrado ser clave para lograr una mayor trazabilidad, precisión en los registros y eficiencia operativa. Por este motivo, cada vez más instituciones están dejando atrás los documentos en papel, las planillas de Excel y los procedimientos dispersos, y están migrando hacia plataformas web que centralizan y organizan la información de forma más segura y confiable. Esta tendencia está ampliamente respaldada por estudios académicos y por la práctica común de centros de investigación y laboratorios en todo el mundo.

\cite{inia-uy,ista2023,fao-seed}

El proyecto desarrollado para el INIA se inscribe en esta línea de evolución tecnológica. La institución necesita una herramienta digital que permita gestionar sus análisis de forma integral: registrar y consultar muestras (lotes), administrar configuraciones, controlar distintos roles de usuario, generar reportes, exportar información y también incorporar datos históricos provenientes de sistemas anteriores. Este capítulo presenta el marco conceptual, tecnológico y comparativo que justifica la solución propuesta y contextualiza su diseño dentro de las tendencias actuales de la industria del software.

\cite{inia-uy}
\subsection{Marco Teórico}
En estos últimos años las organizaciones adoptaron hojas de cálculo como herramienta base para almacenar información gracias a su accesibilidad, bajo costo, flexibilidad y facilidad de uso. Sin embargo, a medida que los procesos se vuelven más complejos y los volúmenes de datos crecen, estos mecanismos se vuelven insuficientes y arriesgados para labores críticas. Las limitaciones más comunes incluyen:
\begin{itemize}
    \item \textbf{Escalabilidad restringida:} Las hojas de cálculo no están diseñadas para manejar grandes volúmenes de datos ni crecer de forma sostenible. A medida que aumentan los registros, las operaciones se vuelven lentas, propensas a fallos y difíciles de mantener.
    \item \textbf{Colaboración limitada:} La funcionalidad de edición simultánea que es ofrecida por la mayoría de las aplicaciones de gestión de hojas de cálculo es vulnerable a conflictos, sobreescrituras y pérdida de información.
    \item \textbf{Falta de trazabilidad y mecanismos de auditoría:} Resulta difícil rastrear cambios, identificar responsables y asegurar integridad de datos.
    \item \textbf{Integración deficiente con sistemas externos:} La conexión con APIs, bases de datos, servicios externos u otros sistemas institucionales resulta limitada.
\end{itemize}
\subsection{La Transición Hacia Sistemas Web}
Las planillas suelen funcionar bien en etapas iniciales, pero a medida que crecen los volúmenes de información o los procesos se vuelven más complejos, es común que se busquen alternativas web que permitan centralizar la información y manejarla de manera más eficiente. Los sistemas modernos ofrecen:
\begin{itemize}
    \item Acceso multiplataforma.
    \item Escalabilidad horizontal mediante arquitecturas distribuidas.
    \item Integración nativa a través de APIs REST.
    \item Validaciones automáticas a nivel de negocio.
    \item Auditorías completas de operaciones.
    \item Estandarización de procesos y flujos de trabajo.
\end{itemize}
\cite{ista2023,fao-seed}
Este enfoque tecnológico es el más utilizado en laboratorios, centros de investigación y organizaciones científicas para gestionar muestras, controles de calidad, análisis y trazabilidad de procesos experimentales.
\subsection{Tecnologías}
El proyecto adopta tecnologías modernas, robustas y ampliamente utilizadas en la industria. La selección busca equilibrio entre estabilidad, madurez, facilidad de mantenimiento y alineación con estándares profesionales.
\cite{spring-boot,nextjs}
\subsubsection{Frontend: React + TypeScript}
React es uno de los frameworks más utilizados globalmente para construir interfaces de usuario debido a:
\begin{itemize}
    \item Su modelo de componentes reutilizables
    \item Su alto rendimiento mediante virtual DOM
    \item Un ecosistema amplio de librerías
    \item Facilidad para construir SPA (Single Page Applications)
\end{itemize}
El uso de TypeScript aporta tipado estático, reduce errores y mejora la mantenibilidad del código.
\subsubsection{Backend: Java + Spring Boot}
Spring Boot es estándar en el desarrollo empresarial gracias a:
\begin{itemize}
    \item Integración nativa con Spring Security
    \item Soporte simplificado para APIs REST
    \item Inyección de dependencias y modularidad
    \item Comunidad madura y documentación extensa
\end{itemize}
Java 21, como versión LTS, asegura estabilidad a largo plazo.
\subsubsection{Base de Datos: PostgreSQL}
Elegida por ser:
\begin{itemize}
    \item Open source
    \item Altamente confiable
    \item Compatible con operaciones complejas
    \item Escalable y sólida para manejo de datos institucionales
\end{itemize}
\subsubsection{Infraestructura y Arquitectura}
\begin{itemize}
    \item Modelo cliente-servidor (frontend Next.js – backend Spring Boot)
    \item Arquitectura en tres capas (presentación – servicios – datos)
    \item Uso de Docker para portabilidad y despliegue
    \item Estructura modular por dominios (análisis, seguridad, notificaciones, usuarios)
    \item Comunicación vía REST y WebSocket (notificaciones en tiempo real) \cite{rfc6455,rfc7519}
\end{itemize}
\subsection{Soluciones Similares}
En el mercado existen herramientas orientadas a gestionar información estructurada, colaborar en equipos y reemplazar flujos basados en planillas. Aunque ninguna se adapta exactamente a los procesos complejos del INIA, sirven como referencia sobre cómo la industria resuelve problemas similares.
\subsubsection{Airtable}
Airtable combina conceptos de base de datos con la interfaz amigable de una hoja de cálculo. Es una plataforma low-code orientada a pequeños proyectos y equipos que necesitan digitalizar procesos sin desarrollar software propio.

Ventajas:
\begin{itemize}
    \item Interfaz simple y accesible
    \item Colaboración en tiempo real
    \item APIs integradas
    \item Automatizaciones básicas
\end{itemize}
Limitaciones:
\begin{itemize}
    \item No escala para flujos complejos
    \item Restricciones para reglas de negocio avanzadas
    \item Dependencia de licencias externas
\end{itemize}
\cite{airtable}
\begin{figure}[htbp]
    \centering
    \includegraphics[width=0.8\textwidth,height=0.6\textheight,keepaspectratio]{Airtable.png}
    \caption{Ejemplo: vista de contenido en Airtable (tabla/registro).}
    \label{fig:airtable}
\end{figure}
\subsubsection{Smartsheet}
Smartsheet ofrece una experiencia similar a Excel, pero con mayor control, trazabilidad y herramientas para flujos de trabajo.

Ventajas:
\begin{itemize}
    \item Gestión de proyectos y procesos
    \item Reportes avanzados
    \item Automatizaciones integradas
\end{itemize}
Limitaciones:
\begin{itemize}
    \item Alto costo según uso
    \item Menor flexibilidad frente a un desarrollo a medida
    \item Dependencia del ecosistema propietario
\end{itemize}
\cite{smartsheet}
\begin{figure}[htbp]
    \centering
    \includegraphics[width=0.8\textwidth,height=0.6\textheight,keepaspectratio]{Smartsheet.png}
    \caption{Ejemplo: interfaz de Smartsheet mostrando una vista tipo hoja de cálculo y reportes.}
    \label{fig:smartsheet}
\end{figure}
\subsubsection{Odoo}
Odoo es un ERP (Planificador de Recursos Empresariales) modular que integra distintos dominios empresariales.

Ventajas:
\begin{itemize}
    \item Gran variedad de módulos
    \item Comunidad activa
    \item Escalabilidad para múltiples áreas
\end{itemize}
Limitaciones:
\begin{itemize}
    \item Instalación y configuración complejas
    \item Sobrecarga funcional para proyectos específicos
    \item Dificultad de adaptación a procesos científicos y de laboratorio
\end{itemize}
\cite{odoo}
\begin{figure}[htbp]
    \centering
    \includegraphics[width=0.8\textwidth,height=0.6\textheight,keepaspectratio]{Odoo.png}
    \caption{Ejemplo: módulo de Odoo con vistas integradas de gestión de recursos.}
    \label{fig:odoo}
\end{figure}
\subsection{Conclusión del análisis comparativo}
Si bien las herramientas ofrecen funcionalidades útiles, no cumplen de forma precisa con los requerimientos del INIA, especialmente en lo relativo a:
\begin{itemize}
    \item Procesos de laboratorio
    \item Gestión de análisis y múltiples roles
    \item Validaciones específicas
    \item Importación de datos legados
    \item Adaptación a flujos técnicos
\end{itemize}
Por ello, un sistema a medida es la opción más adecuada para atender las necesidades reales y garantizar la adaptación total a los procesos internos y ofrecer la flexibilidad necesaria para futuros cambios o expansiones.
\cite{inia-uy,ista2023}