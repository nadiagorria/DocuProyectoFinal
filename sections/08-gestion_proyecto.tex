\section{Gestión de Proyecto}
Para la ejecución de este proyecto no se estructuraron roles específicos para cada integrante del equipo. Si bien cada miembro tenía mayor afinidad con determinadas áreas tecnológicas, se decidió adoptar una dinámica de trabajo colaborativa en la que todos participaron tanto en el desarrollo del frontend como el backend, así como en tareas vinculadas a documentación, testing y despliegue.
\cite{inia-uy}

Para la organización del trabajó se creó un proyecto en Notion, donde se registraron las tareas principales, los responsables y las estimaciones de tiempo para cada tarea. 
\cite{notion}

El principal medio de comunicación en línea fue Google Meet, utilizado para reuniones virtuales, donde a su vez se llevaron a cabo reuniones periódicas con el cliente. Además, se creó un grupo de WhatsApp para coordinar rápidamente temas urgentes, notificaciones internas y actualizaciones cortas entre los integrantes del equipo, lo que permitió una comunicación ágil y continua.
\cite{googlemeet,whatsapp}
\subsection{Entorno de Desarrollo}
El desarrollo del sistema se realizó en un entorno orientado a aplicaciones web modernas, priorizando la reproducibilidad, la trazabilidad y la correcta separación de responsabilidades entre los distintos componentes.
\cite{docker}

Para el desarrollo del software se utilizaron dos entornos de desarrollo integrados (IDE):
\begin{itemize}
    \item Visual Studio Code, empleado principalmente para el frontend y configurado con extensiones específicas para TypeScript, React, Docker y herramientas de control de versiones.
    \cite{vscode}
    \item IntelliJ IDEA, utilizado para el backend en Java, aprovechando su soporte avanzado para Spring Boot, gestión de dependencias con Maven y depuración integrada.
    \cite{intellij}
\end{itemize}
La gestión del código fuente se realizó utilizando Git como sistema de control de versiones y GitHub como plataforma principal de alojamiento, revisión y seguimiento del proyecto. GitHub permitió manejar ramas, realizar revisiones de código, gestionar issues y mantener un flujo de trabajo organizado basado en buenas prácticas de versionado.
\cite{git,github}

Esta combinación permitió a todos los integrantes del equipo trabajar de forma eficiente y con herramientas adecuadas para cada tecnología.

\vspace{1em}
\noindent\textbf{Nota:} Para información detallada sobre la identificación, evaluación y mitigación de riesgos del proyecto, consultar el \textbf{Anexo 3 -- Mitigación de Riesgos} en el documento \texttt{Anexos.pdf}.
