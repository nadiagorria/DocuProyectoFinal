\section{Problemas Encontrados}
\subsection{Reestructuración del Módulo de Germinación}
Durante la etapa final del desarrollo se identificó un problema crítico en el módulo de germinación, que obligó a replantear su estructura, lógica interna y funcionamiento general. Este inconveniente surgió a raíz de una discrepancia significativa entre los requisitos originalmente planteados y las necesidades reales del laboratorio\cite{inia-uy,ista2023}.

La dificultad se originó en la documentación inicial proporcionada por el cliente, compuesta por hojas de cálculo con información incompleta, las cuales eran utilizadas por el equipo para realizar la gestión de forma manual. A partir de la referencia el equipo asumió que todas las fechas de conteo eran comunes a todo el análisis de germinación y que las diferentes configuraciones de días de prefrío ingresadas en el mismo análisis no influían en las mismas, entre otras\cite{fao-seed,ista2023}.

Estas suposiciones llevaron a diseñar el módulo bajo una estructura que no reflejaba la complejidad real del proceso.
\subsubsection{Cambio de Requisitos}
Pocos días antes de la finalización de la etapa de desarrollo, el cliente aclaró que el funcionamiento real del análisis de germinación difería de lo inicialmente interpretado. Los requisitos correctos incluían configuraciones múltiples con fechas de conteos independientes, restringidas por los días de prefrío y con validaciones específicas adicionales.

Este cambio presentó un ajuste conceptual profundo respecto a la estructura inicial del módulo, y su corrección requirió una reingeniería completa del módulo, afectando múltiples áreas del sistema. Entre las acciones necesarias se incluyó el rediseño de la estructura de los datos, reescritura de la lógica del sistema y modificación de interfaces.

A pesar de las restricciones de tiempo, el equipo logró implementar correctamente la nueva estructura, la cual resultó ser fundamental para asegurar que el módulo cumpliera adecuadamente con los cambios solicitados por el cliente\cite{inia-uy}.