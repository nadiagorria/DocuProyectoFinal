\section{La Solución Desarrollada}
El Sistema de Gestión de Análisis de Semillas desarrollado para INIA es una plataforma web integral que digitaliza y centraliza todo el proceso de análisis de calidad de semillas, reemplazando el manejo fragmentado que anteriormente se realizaba en múltiples planillas de Excel. La solución permite registrar lotes que ingresan al laboratorio, realizar distintos tipos de análisis especializados y generar reportes estandarizados para INIA e INASE (Instituto Nacional de Semillas) con mayor precisión, trazabilidad y control\cite{inia-uy,spring-boot,nextjs}.
\subsection{Objetivos de la Solución}
La solución desarrollada busca digitalizar y unificar el trabajo del laboratorio de semillas. En particular, se propuso:
\begin{itemize}
    \item Centralizar la información de todos los análisis en un único sistema accesible desde cualquier dispositivo.
    \item Estandarizar el registro de los lotes de semillas y de los resultados obtenidos en cada análisis.
    \item Reducir errores eliminando la manipulación manual de múltiples planillas.
    \item Aumentar la trazabilidad mediante estados de avance y registro estructurado de datos.
    \item Mejorar la eficiencia del laboratorio, automatizando validaciones, cálculos y generación de reportes.
    \item Facilitar el cumplimiento normativo, generando reportes compatibles con los requerimientos de INIA e INASE.
    \item Asegurar el acceso controlado, mediante roles de usuario y autenticación JWT\cite{rfc7519}.
\end{itemize}
\subsubsection{Gestión de Lotes de Semillas}
El sistema soporta los análisis de Pureza Física, Germinación, DOSN, PMS y Tetrazolio, permitiendo:
\begin{itemize}
    \item Carga estructurada de resultados y repeticiones.
    \item Cálculos automáticos (porcentajes, promedios, germinación final, etc.).
    \item Validaciones básicas según normas vigentes y señalización de valores críticos.
\end{itemize}
\subsubsection{Registro de análisis de calidad}
El sistema permite realizar distintos análisis obligatorios en los laboratorios de semillas:
\begin{itemize}
    \item Registro de porcentajes de semilla pura, materia inerte, otros cultivos y malezas.
    \item Cálculos automáticos de proporciones y verificaciones según normas vigentes.
\end{itemize}
\subsubsection{Flujo de Trabajo del Laboratorio}
\begin{itemize}
    \item Manejo de estados de los análisis (EN PROCESO, FINALIZADO, APROBADO).
    \item Restricción de acciones según el rol del usuario.
    \item Registro del historial de cambios y validaciones realizadas.
\end{itemize}
\subsubsection{Reportes y Exportaciones}
\begin{itemize}
    \item Generación automática de reportes en Excel con todas las columnas requeridas por INIA e INASE.
    \item Integración de datos de lote, resultados y observaciones en un formato estandarizado para auditorías y trazabilidad\cite{openapi,postgresql,docker}.
\end{itemize}
En síntesis, la solución implementada cumplió con los objetivos planteados: centralizar la información del laboratorio, estandarizar el registro de lotes y análisis, mejorar la trazabilidad y facilitar el cumplimiento normativo frente a INIA e INASE.

A continuación se presentan capturas de pantalla del sistema que ilustran la carga de lotes, el flujo de trabajo de los análisis y los reportes generados.

\begin{figure}[htbp]
    \centering
    \includegraphics[width=0.85\textwidth]{registroDeLote.png}
    \caption{Pantalla de Registro de Lotes: permite crear y editar la ficha de cada lote, asociar cultivares y empresas, y consultar los últimos lotes ingresados.}
    \label{fig:registro-lote}
\end{figure}

\begin{figure}[htbp]
    \centering
    \includegraphics[width=0.85\textwidth]{ListadoDeLote.png}
    \caption{Pantalla de Listado de Lotes: búsqueda, filtros y acceso al detalle de cada lote y sus análisis asociados.}
    \label{fig:listado-lotes}
\end{figure}

\begin{figure}[htbp]
    \centering
    \includegraphics[width=0.85\textwidth]{RegistroDeAnalisis.png}
    \caption{Pantalla de Registro de Análisis: selección del tipo de análisis (Pureza Física, Germinación, DOSN, PMS, Tetrazolio) y asociación al lote correspondiente.}
    \label{fig:registro-analisis}
\end{figure}

\begin{figure}[htbp]
    \centering
    \includegraphics[width=0.85\textwidth]{CentroDeReportes.png}
    \caption{Centro de Reportes: generación y exportación de reportes a Excel con filtros configurables.}
    \label{fig:centro-reportes}
\end{figure}

\begin{figure}[htbp]
    \centering
    \includegraphics[width=0.85\textwidth]{PanelDeMetricas1.png}
    \caption{Panel de métricas (1): indicadores de análisis por período.}
    \label{fig:panel-metricas-1}
\end{figure}

\begin{figure}[htbp]
    \centering
    \includegraphics[width=0.85\textwidth]{PanelDeMetricas2.png}
    \caption{Panel de métricas (2): distribución por tipo y estado.}
    \label{fig:panel-metricas-2}
\end{figure}

% Nota: las imágenes provienen de la carpeta `images/` del documento principal.