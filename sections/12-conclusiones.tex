\section{Conclusiones}
El desarrollo del Sistema de Gestión de Laboratorio de Semillas (SGLS) para INIA Uruguay representa un avance clave en la modernización de los procesos de control de calidad. El proyecto surge para sustituir flujos de trabajo manuales basados en planillas independientes, que generaban dispersión de datos, dificultades en la trazabilidad y una alta dependencia del registro humano. Estas limitaciones afectaban la eficiencia diaria del laboratorio y podían comprometer la confiabilidad de los resultados, por lo que avanzar hacia un sistema centralizado y digital se volvió una necesidad estratégica.
Con la implementación del SGLS, se logró integrar en un único entorno todas las etapas del análisis de semillas, desde la recepción de muestras hasta la emisión de informes finales. Este enfoque unificado permitió eliminar la duplicación de datos, mejorar la trazabilidad de cada lote y asegurar un seguimiento claro y preciso a lo largo de todo el proceso. La automatización de cálculos complejos y la validación de información redujo considerablemente los errores humanos, aportando mayor consistencia y confianza en los resultados generados. Esto no solo agiliza el trabajo interno, sino que también fortalece la calidad de los servicios ofrecidos por el laboratorio.
Finalmente, el sistema no solo cumple con los requerimientos actuales del proceso de certificación, sino que establece una base sólida para el crecimiento futuro. Su diseño organizado y adaptable permite incorporar nuevos módulos, integrar equipamiento automatizado y ampliar las capacidades analíticas y de reporte que el laboratorio pueda necesitar más adelante. En conjunto, el SGLS posiciona a INIA como una institución alineada con las demandas modernas del sector agropecuario, demostrando cómo la digitalización puede potenciar la productividad, garantizar la calidad y respaldar la toma de decisiones con datos confiables.