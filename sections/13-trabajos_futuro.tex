\section{Trabajo a Futuro}
\subsection{Módulo de auditorías y control de calidad}
Si bien el sistema registra información detallada de cada análisis, podría incorporarse un módulo específico para auditorías externas e internas más específicas. Este módulo permitiría revisar trazabilidad histórica, validar procedimientos y generar informes automáticos para procesos de certificación, algo especialmente útil cuando se trabaja en conjunto con INASE u otras instituciones reguladoras.
\cite{ista2023,fao-seed,inia-uy}
\subsection{Gestión documental avanzada}
Actualmente se generan reportes en Excel, pero no existe un repositorio interno de documentos. A futuro podría añadirse un módulo que permita almacenar protocolos, fotos de muestras, resultados complementarios, certificados y archivos adjuntos por lote. Esto facilitaría consultas históricas y el intercambio de información entre analistas.
\cite{notion}
\subsection{Automatización de cálculos y validaciones normativas}
Aunque el sistema ya automatiza buena parte de los cálculos, todavía existen validaciones que podrían formalizarse según normativas específicas de INASE o reglas internacionales de análisis de semillas. Incluir este módulo permitiría evitar inconsistencias y reducir posibles errores humanos en los análisis más complejos.
\cite{ista2023,fao-seed}
\subsection{Optimización del flujo de trabajo}
Actualmente el sistema maneja estados básicos de los análisis (REGISTRADO (en algunos casos) \textrightarrow{} EN\_PROCESO \textrightarrow{} FINALIZADO \textrightarrow{} APROBADO). A futuro podría añadirse un motor más completo de \textit{workflow} que permita definir etapas personalizadas, responsables por análisis, plazos estimados y automatización de pasos repetitivos. Esto ayudaría a estandarizar procesos entre distintos laboratorios.
\cite{notion,openapi}
\subsection{Integración con sistemas externos}
En una fase posterior podría evaluarse la integración con sistemas institucionales externos utilizados por INIA o INASE, como portales de registro, bases de datos oficiales o plataformas de trazabilidad. Esto reduciría la duplicación de información y agilizaría los procesos administrativos.
\cite{openapi,postgresql}